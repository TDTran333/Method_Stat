\documentclass[]{article}
\usepackage{lmodern}
\usepackage{amssymb,amsmath}
\usepackage{ifxetex,ifluatex}
\usepackage{fixltx2e} % provides \textsubscript
\ifnum 0\ifxetex 1\fi\ifluatex 1\fi=0 % if pdftex
  \usepackage[T1]{fontenc}
  \usepackage[utf8]{inputenc}
\else % if luatex or xelatex
  \ifxetex
    \usepackage{mathspec}
  \else
    \usepackage{fontspec}
  \fi
  \defaultfontfeatures{Ligatures=TeX,Scale=MatchLowercase}
\fi
% use upquote if available, for straight quotes in verbatim environments
\IfFileExists{upquote.sty}{\usepackage{upquote}}{}
% use microtype if available
\IfFileExists{microtype.sty}{%
\usepackage{microtype}
\UseMicrotypeSet[protrusion]{basicmath} % disable protrusion for tt fonts
}{}
\usepackage[margin=1in]{geometry}
\usepackage{hyperref}
\hypersetup{unicode=true,
            pdftitle={Statistical Methods in Financial Engineering - Risk Management Project},
            pdfauthor={Chloe Morin-Leclerc, Felix-Antoine Groulx, Denis Genest, Thien Duy Tran},
            pdfborder={0 0 0},
            breaklinks=true}
\urlstyle{same}  % don't use monospace font for urls
\usepackage{color}
\usepackage{fancyvrb}
\newcommand{\VerbBar}{|}
\newcommand{\VERB}{\Verb[commandchars=\\\{\}]}
\DefineVerbatimEnvironment{Highlighting}{Verbatim}{commandchars=\\\{\}}
% Add ',fontsize=\small' for more characters per line
\usepackage{framed}
\definecolor{shadecolor}{RGB}{248,248,248}
\newenvironment{Shaded}{\begin{snugshade}}{\end{snugshade}}
\newcommand{\AlertTok}[1]{\textcolor[rgb]{0.94,0.16,0.16}{#1}}
\newcommand{\AnnotationTok}[1]{\textcolor[rgb]{0.56,0.35,0.01}{\textbf{\textit{#1}}}}
\newcommand{\AttributeTok}[1]{\textcolor[rgb]{0.77,0.63,0.00}{#1}}
\newcommand{\BaseNTok}[1]{\textcolor[rgb]{0.00,0.00,0.81}{#1}}
\newcommand{\BuiltInTok}[1]{#1}
\newcommand{\CharTok}[1]{\textcolor[rgb]{0.31,0.60,0.02}{#1}}
\newcommand{\CommentTok}[1]{\textcolor[rgb]{0.56,0.35,0.01}{\textit{#1}}}
\newcommand{\CommentVarTok}[1]{\textcolor[rgb]{0.56,0.35,0.01}{\textbf{\textit{#1}}}}
\newcommand{\ConstantTok}[1]{\textcolor[rgb]{0.00,0.00,0.00}{#1}}
\newcommand{\ControlFlowTok}[1]{\textcolor[rgb]{0.13,0.29,0.53}{\textbf{#1}}}
\newcommand{\DataTypeTok}[1]{\textcolor[rgb]{0.13,0.29,0.53}{#1}}
\newcommand{\DecValTok}[1]{\textcolor[rgb]{0.00,0.00,0.81}{#1}}
\newcommand{\DocumentationTok}[1]{\textcolor[rgb]{0.56,0.35,0.01}{\textbf{\textit{#1}}}}
\newcommand{\ErrorTok}[1]{\textcolor[rgb]{0.64,0.00,0.00}{\textbf{#1}}}
\newcommand{\ExtensionTok}[1]{#1}
\newcommand{\FloatTok}[1]{\textcolor[rgb]{0.00,0.00,0.81}{#1}}
\newcommand{\FunctionTok}[1]{\textcolor[rgb]{0.00,0.00,0.00}{#1}}
\newcommand{\ImportTok}[1]{#1}
\newcommand{\InformationTok}[1]{\textcolor[rgb]{0.56,0.35,0.01}{\textbf{\textit{#1}}}}
\newcommand{\KeywordTok}[1]{\textcolor[rgb]{0.13,0.29,0.53}{\textbf{#1}}}
\newcommand{\NormalTok}[1]{#1}
\newcommand{\OperatorTok}[1]{\textcolor[rgb]{0.81,0.36,0.00}{\textbf{#1}}}
\newcommand{\OtherTok}[1]{\textcolor[rgb]{0.56,0.35,0.01}{#1}}
\newcommand{\PreprocessorTok}[1]{\textcolor[rgb]{0.56,0.35,0.01}{\textit{#1}}}
\newcommand{\RegionMarkerTok}[1]{#1}
\newcommand{\SpecialCharTok}[1]{\textcolor[rgb]{0.00,0.00,0.00}{#1}}
\newcommand{\SpecialStringTok}[1]{\textcolor[rgb]{0.31,0.60,0.02}{#1}}
\newcommand{\StringTok}[1]{\textcolor[rgb]{0.31,0.60,0.02}{#1}}
\newcommand{\VariableTok}[1]{\textcolor[rgb]{0.00,0.00,0.00}{#1}}
\newcommand{\VerbatimStringTok}[1]{\textcolor[rgb]{0.31,0.60,0.02}{#1}}
\newcommand{\WarningTok}[1]{\textcolor[rgb]{0.56,0.35,0.01}{\textbf{\textit{#1}}}}
\usepackage{graphicx,grffile}
\makeatletter
\def\maxwidth{\ifdim\Gin@nat@width>\linewidth\linewidth\else\Gin@nat@width\fi}
\def\maxheight{\ifdim\Gin@nat@height>\textheight\textheight\else\Gin@nat@height\fi}
\makeatother
% Scale images if necessary, so that they will not overflow the page
% margins by default, and it is still possible to overwrite the defaults
% using explicit options in \includegraphics[width, height, ...]{}
\setkeys{Gin}{width=\maxwidth,height=\maxheight,keepaspectratio}
\IfFileExists{parskip.sty}{%
\usepackage{parskip}
}{% else
\setlength{\parindent}{0pt}
\setlength{\parskip}{6pt plus 2pt minus 1pt}
}
\setlength{\emergencystretch}{3em}  % prevent overfull lines
\providecommand{\tightlist}{%
  \setlength{\itemsep}{0pt}\setlength{\parskip}{0pt}}
\setcounter{secnumdepth}{0}
% Redefines (sub)paragraphs to behave more like sections
\ifx\paragraph\undefined\else
\let\oldparagraph\paragraph
\renewcommand{\paragraph}[1]{\oldparagraph{#1}\mbox{}}
\fi
\ifx\subparagraph\undefined\else
\let\oldsubparagraph\subparagraph
\renewcommand{\subparagraph}[1]{\oldsubparagraph{#1}\mbox{}}
\fi

%%% Use protect on footnotes to avoid problems with footnotes in titles
\let\rmarkdownfootnote\footnote%
\def\footnote{\protect\rmarkdownfootnote}

%%% Change title format to be more compact
\usepackage{titling}

% Create subtitle command for use in maketitle
\providecommand{\subtitle}[1]{
  \posttitle{
    \begin{center}\large#1\end{center}
    }
}

\setlength{\droptitle}{-2em}

  \title{Statistical Methods in Financial Engineering - Risk Management
Project}
    \pretitle{\vspace{\droptitle}\centering\huge}
  \posttitle{\par}
    \author{Chloe Morin-Leclerc, Felix-Antoine Groulx, Denis Genest,
Thien Duy Tran}
    \preauthor{\centering\large\emph}
  \postauthor{\par}
      \predate{\centering\large\emph}
  \postdate{\par}
    \date{December 14, 2019}


\begin{document}
\maketitle

\hypertarget{loom-video-link-httpswww.loom.comshare64a6aa9e2f914be494eeae7106631811}{%
\section{\texorpdfstring{LOOM video link:
\url{https://www.loom.com/share/64a6aa9e2f914be494eeae7106631811}}{LOOM video link: https://www.loom.com/share/64a6aa9e2f914be494eeae7106631811}}\label{loom-video-link-httpswww.loom.comshare64a6aa9e2f914be494eeae7106631811}}

\hypertarget{part-i-project-guidelines}{%
\section{Part I: Project Guidelines}\label{part-i-project-guidelines}}

\hypertarget{context}{%
\subsection{Context}\label{context}}

You work as a quantitative analyst for a large investment bank. You and
your team are responsible for challenging the models used by traders and
risk managers. You work with R and love reproducible research. All your
files are written in Rmarkdown or R notebook.

You can watch the introductory video on Rmarkdown to help you build
properly the R file. Moreover, you use GitHub with your team. You'll
build a dedicated project for the tasks below and use RStudio with
GitHub extensively. You also use Loom to share your findings with other
teams in the investment bank.

\hypertarget{learning-objectives}{%
\subsection{Learning Objectives}\label{learning-objectives}}

\begin{enumerate}
\def\labelenumi{\arabic{enumi}.}
\tightlist
\item
  Content (scientific rigor, concepts, creativity).\\
\item
  Choose the right tools.\\
\item
  Implement the steps correctly.\\
\item
  Come up with innovative solutions.\\
\end{enumerate}

\hypertarget{form-coding-collaboration-and-presentation}{%
\subsection{Form: Coding, Collaboration and
Presentation}\label{form-coding-collaboration-and-presentation}}

\begin{enumerate}
\def\labelenumi{\arabic{enumi}.}
\tightlist
\item
  Build RStudio project with proper folder structure and
  Rmarkdown/nootebook file to reproduce your results.\\
\item
  Program with state-of-the-art coding standards.\\
\item
  Use GitHub repository for collaborative research.\\
\item
  Use Loom video for presenting your results.
\end{enumerate}

\hypertarget{objective}{%
\subsection{Objective}\label{objective}}

The objective of this project is to implement the risk management
framework used for estimating the risk of a book of European call
options by taking into account risk drivers such as the underlying asset
and the implied volatility of the options.

\hypertarget{part-ii-data}{%
\section{Part II: Data}\label{part-ii-data}}

\hypertarget{loading-the-data}{%
\subsection{Loading the Data}\label{loading-the-data}}

The first step is to load the database `Market'. `Market' is a list of 5
elements: S\&P500 index prices, VIX values, the term structure of
interest rates, and traded call and put options information. To make
sure this code can run on any platform, we use the library `here'. Also,
we load 2 functions, the first one is use to price options using
Black-Scholes and the second one is use to interpolate the term
structure of interest rates. These functions will be detailled further
in this script.

\begin{Shaded}
\begin{Highlighting}[]
\CommentTok{\# install.packages("here")}
\KeywordTok{library}\NormalTok{(}\StringTok{"here"}\NormalTok{)}
\end{Highlighting}
\end{Shaded}

\begin{verbatim}
## here() starts at C:/Users/tdtra/Desktop/Method Stat/New Project/Method_Stat
\end{verbatim}

\begin{Shaded}
\begin{Highlighting}[]
\CommentTok{\# Load the data}
\KeywordTok{load}\NormalTok{(}\DataTypeTok{file =} \KeywordTok{here}\NormalTok{(}\StringTok{"Data"}\NormalTok{, }\StringTok{"Market.rda"}\NormalTok{))}

\CommentTok{\# Load the functions}
\KeywordTok{source}\NormalTok{(}\DataTypeTok{file =} \KeywordTok{here}\NormalTok{(}\StringTok{"Functions"}\NormalTok{, }\StringTok{"price\_BS.r"}\NormalTok{)) }\CommentTok{\# Option prices using the Black{-}Scholes formula}
\KeywordTok{source}\NormalTok{(}\DataTypeTok{file =} \KeywordTok{here}\NormalTok{(}\StringTok{"Functions"}\NormalTok{, }\StringTok{"lin\_inter.r"}\NormalTok{))  }\CommentTok{\# Linear interpolation of the interest rates}

\CommentTok{\# Assign data to different variables}
\NormalTok{vix       <{-}}\StringTok{ }\KeywordTok{as.vector}\NormalTok{(Market}\OperatorTok{$}\NormalTok{vix)}
\NormalTok{sp\_}\DecValTok{500}\NormalTok{    <{-}}\StringTok{ }\KeywordTok{as.vector}\NormalTok{(Market}\OperatorTok{$}\NormalTok{sp500)}
\NormalTok{calls     <{-}}\StringTok{ }\KeywordTok{as.vector}\NormalTok{(Market}\OperatorTok{$}\NormalTok{calls)}
\NormalTok{puts      <{-}}\StringTok{ }\KeywordTok{as.vector}\NormalTok{(Market}\OperatorTok{$}\NormalTok{puts)}

\CommentTok{\# Create a matrix \textquotesingle{}rates\textquotesingle{} with two columns: maturities and risk{-}free rates}
\NormalTok{rates     <{-}}\StringTok{ }\KeywordTok{matrix}\NormalTok{(}\DataTypeTok{data =} \OtherTok{NA}\NormalTok{, }\DataTypeTok{nrow =} \KeywordTok{length}\NormalTok{(Market}\OperatorTok{$}\NormalTok{rf), }\DataTypeTok{ncol =} \DecValTok{2}\NormalTok{)}
\NormalTok{rates[,}\DecValTok{1}\NormalTok{] <{-}}\StringTok{ }\KeywordTok{as.numeric}\NormalTok{((}\KeywordTok{attributes}\NormalTok{(Market}\OperatorTok{$}\NormalTok{rf))[[}\DecValTok{2}\NormalTok{]])}
\NormalTok{rates[,}\DecValTok{2}\NormalTok{] <{-}}\StringTok{ }\NormalTok{Market}\OperatorTok{$}\NormalTok{rf}

\CommentTok{\# Assign column names to \textquotesingle{}rates\textquotesingle{}}
\KeywordTok{colnames}\NormalTok{(rates) <{-}}\StringTok{ }\KeywordTok{c}\NormalTok{(}\StringTok{"Maturities"}\NormalTok{, }\StringTok{"Risk{-}free Rates"}\NormalTok{)}
\end{Highlighting}
\end{Shaded}

\hypertarget{part-iii-pricing-a-portfolio-of-options}{%
\section{Part III: Pricing a Portfolio of
Options}\label{part-iii-pricing-a-portfolio-of-options}}

The portfolio under consideration contains four options: K = 1600 and
T-t = 20 days, K = 1650 and T-t = 20 days, K = 1750 and T-t = 40 days,
and K = 1800 and T-t = 40 days. K is the strike price and T-t is days
before maturity. We assume that there is 250 days in a year and thus
convert times to expiry in years by dividing by 250. We first create a
matrix `book' that contains information regarding the options in the
portfolio.

\begin{Shaded}
\begin{Highlighting}[]
\CommentTok{\# Create matrix}
\NormalTok{book <{-}}\StringTok{ }\KeywordTok{matrix}\NormalTok{(}\DataTypeTok{data =} \OtherTok{NA}\NormalTok{, }\DataTypeTok{nrow =} \DecValTok{4}\NormalTok{, }\DataTypeTok{ncol =} \DecValTok{6}\NormalTok{)}

\CommentTok{\# Assign names to columns}
\CommentTok{\# Option Type: Call = 1, Put = 0}
\KeywordTok{colnames}\NormalTok{(book) <{-}}\StringTok{ }\KeywordTok{c}\NormalTok{(}\StringTok{"Quantity"}\NormalTok{, }\StringTok{"Option Type"}\NormalTok{, }\StringTok{"Strike Price"}\NormalTok{, }\StringTok{"Maturity"}\NormalTok{, }\StringTok{"Option price"}\NormalTok{, }\StringTok{"Position Value"}\NormalTok{)}

\CommentTok{\# Store initial values}
\NormalTok{book[}\DecValTok{1}\NormalTok{,}\DecValTok{1}\OperatorTok{:}\DecValTok{4}\NormalTok{] <{-}}\StringTok{ }\KeywordTok{c}\NormalTok{(}\DecValTok{1}\NormalTok{, }\DecValTok{1}\NormalTok{, }\DecValTok{1600}\NormalTok{, }\DecValTok{20} \OperatorTok{/}\StringTok{ }\DecValTok{250}\NormalTok{)}
\NormalTok{book[}\DecValTok{2}\NormalTok{,}\DecValTok{1}\OperatorTok{:}\DecValTok{4}\NormalTok{] <{-}}\StringTok{ }\KeywordTok{c}\NormalTok{(}\DecValTok{1}\NormalTok{, }\DecValTok{1}\NormalTok{, }\DecValTok{1650}\NormalTok{, }\DecValTok{20} \OperatorTok{/}\StringTok{ }\DecValTok{250}\NormalTok{)}
\NormalTok{book[}\DecValTok{3}\NormalTok{,}\DecValTok{1}\OperatorTok{:}\DecValTok{4}\NormalTok{] <{-}}\StringTok{ }\KeywordTok{c}\NormalTok{(}\DecValTok{1}\NormalTok{, }\DecValTok{1}\NormalTok{, }\DecValTok{1750}\NormalTok{, }\DecValTok{40} \OperatorTok{/}\StringTok{ }\DecValTok{250}\NormalTok{)}
\NormalTok{book[}\DecValTok{4}\NormalTok{,}\DecValTok{1}\OperatorTok{:}\DecValTok{4}\NormalTok{] <{-}}\StringTok{ }\KeywordTok{c}\NormalTok{(}\DecValTok{1}\NormalTok{, }\DecValTok{1}\NormalTok{, }\DecValTok{1800}\NormalTok{, }\DecValTok{40} \OperatorTok{/}\StringTok{ }\DecValTok{250}\NormalTok{)}
\end{Highlighting}
\end{Shaded}

The next step is to use the most recent underlying asset price and VIX
value as well as the interpolated risk-free rates to compute the price
of each option in the portfolio. The value of the portfolio of options
is simply the sum of each option's price multiplied by its quantity. To
do this, we use two functions described below.

\begin{enumerate}
\def\labelenumi{\arabic{enumi}.}
\tightlist
\item
  `lin\_inter': Takes as input a maturity in years (360-day basis) and
  outputs the associated rate. Uses linear interpolation with the given
  term structure.
\item
  `price\_BS': Takes as inputs the underlying asset price, the option
  strike price, the risk-free rate, the volatility, the maturity and the
  option type. Uses Black-Scholes model to price options.
\end{enumerate}

\begin{Shaded}
\begin{Highlighting}[]
\CommentTok{\# Number of observations}
\NormalTok{n\_obs <{-}}\StringTok{ }\KeywordTok{length}\NormalTok{(sp\_}\DecValTok{500}\NormalTok{)}

\CommentTok{\# Convert 250{-}day basis year in 360{-}day basis year}
\NormalTok{m <{-}}\StringTok{ }\NormalTok{book[,}\DecValTok{4}\NormalTok{]}\OperatorTok{*}\NormalTok{(}\DecValTok{250} \OperatorTok{/}\StringTok{ }\DecValTok{360}\NormalTok{)}

\CommentTok{\# Interpolated interest rates}
\NormalTok{rf <{-}}\StringTok{ }\KeywordTok{mapply}\NormalTok{(lin\_inter, m)}

\CommentTok{\# Latest underlying asset price (spot price)}
\NormalTok{S0 <{-}}\StringTok{ }\NormalTok{sp\_}\DecValTok{500}\NormalTok{[n\_obs]}
\NormalTok{S <{-}}\StringTok{ }\KeywordTok{matrix}\NormalTok{(}\DataTypeTok{data =}\NormalTok{ S0, }\DataTypeTok{nrow =} \DecValTok{4}\NormalTok{)}

\CommentTok{\# Latest VIX value}
\NormalTok{Vol0 <{-}}\StringTok{ }\NormalTok{vix[n\_obs]}
\NormalTok{Vol <{-}}\StringTok{ }\KeywordTok{matrix}\NormalTok{(}\DataTypeTok{data =}\NormalTok{ Vol0, }\DataTypeTok{nrow =} \DecValTok{4}\NormalTok{)}

\CommentTok{\# Strike prices}
\NormalTok{K <{-}}\StringTok{ }\NormalTok{book[,}\DecValTok{3}\NormalTok{]}

\CommentTok{\# Maturities}
\NormalTok{M <{-}}\StringTok{ }\NormalTok{book[,}\DecValTok{4}\NormalTok{]}

\CommentTok{\# Option type}
\NormalTok{Type <{-}}\StringTok{ }\NormalTok{book[,}\DecValTok{2}\NormalTok{]}

\CommentTok{\# Compute the options values}
\NormalTok{book[,}\DecValTok{5}\NormalTok{] <{-}}\StringTok{ }\KeywordTok{mapply}\NormalTok{(price\_BS, S, K, rf, Vol, M, Type)}

\CommentTok{\# Compute the value of the portfolio}
\NormalTok{book[,}\DecValTok{6}\NormalTok{] <{-}}\StringTok{ }\NormalTok{book[,}\DecValTok{1}\NormalTok{]}\OperatorTok{*}\NormalTok{book[,}\DecValTok{5}\NormalTok{]}
\NormalTok{PF\_val   <{-}}\StringTok{ }\KeywordTok{sum}\NormalTok{(book[,}\DecValTok{6}\NormalTok{])}
\end{Highlighting}
\end{Shaded}

\begin{verbatim}
## The total value of the Options Portfolio is 156.98$.
\end{verbatim}

\begin{verbatim}
## 
## Portfolio detail:
\end{verbatim}

\begin{verbatim}
##      Quantity Option Type Strike Price Maturity Option price
## [1,]    1.00     1.00     1600.00         0.08    87.57     
## [2,]    1.00     1.00     1650.00         0.08    47.72     
## [3,]    1.00     1.00     1750.00         0.16    15.30     
## [4,]    1.00     1.00     1800.00         0.16     6.38     
##      Position Value
## [1,]   87.57       
## [2,]   47.72       
## [3,]   15.30       
## [4,]    6.38
\end{verbatim}

It is not surprising to see that the value of the first option is higher
than the value of the second option. Indeed, since the strike price is
lower for the first option, the first option is deeper ITM than the
second option and it should therefore have a higher price. The same
reasoning applies to the third and fourth options.

\hypertarget{part-iv-risk-model-1-one-risk-driver-and-gaussian-distribution}{%
\section{Part IV: Risk Model 1: One Risk Driver and Gaussian
Distribution}\label{part-iv-risk-model-1-one-risk-driver-and-gaussian-distribution}}

We now want to estimate the value-at-risk (VaR) and the expected
shortfall (ES) of this portfolio of options over the course of the
following week. To do so, we must first compute the underlying asset log
returns.

\begin{Shaded}
\begin{Highlighting}[]
\CommentTok{\# Daily underlying asset log returns}
\NormalTok{log\_return <{-}}\StringTok{ }\KeywordTok{diff}\NormalTok{(}\KeywordTok{log}\NormalTok{(sp\_}\DecValTok{500}\NormalTok{))}
\end{Highlighting}
\end{Shaded}

Next, we assume that the underlying asset log returns follow a normal
distribution. The normal distribution parameters can be found by
computing the empirical mean and standard deviation of the underlying
asset daily log returns.

\begin{Shaded}
\begin{Highlighting}[]
\CommentTok{\# Number of log returns}
\NormalTok{n\_ret <{-}}\StringTok{ }\KeywordTok{length}\NormalTok{(log\_return)}

\CommentTok{\# Calibration}
\NormalTok{mu\_hat\_}\DecValTok{1}\NormalTok{ <{-}}\StringTok{ }\KeywordTok{mean}\NormalTok{(log\_return)}
\NormalTok{s2\_hat\_}\DecValTok{1}\NormalTok{ <{-}}\StringTok{ }\KeywordTok{var}\NormalTok{(log\_return)}
\NormalTok{sd\_hat\_}\DecValTok{1}\NormalTok{  <{-}}\StringTok{ }\NormalTok{s2\_hat\_}\DecValTok{1}\OperatorTok{\^{}}\FloatTok{0.5}

\CommentTok{\# Store parameters in \textquotesingle{}theta\_1\textquotesingle{}}
\NormalTok{theta\_}\DecValTok{1}\NormalTok{ <{-}}\StringTok{ }\KeywordTok{c}\NormalTok{(mu\_hat\_}\DecValTok{1}\NormalTok{, sd\_hat\_}\DecValTok{1}\NormalTok{)}
\end{Highlighting}
\end{Shaded}

Now that we have estimated the parameters of the underlying asset return
log returns, we can run a simulation of the underlying asset price one
week ahead by generating normally distributed IID shocks with mean
`mu\_hat' and standard deviation `sd\_hat'. Since we found the
distribution parameters using daily log returns, we generate five shocks
per simulation and sum these shocks to obtain the overall shock over one
week. We fix the size of the simulation to 10 000.

\begin{Shaded}
\begin{Highlighting}[]
\CommentTok{\# Number of simulation}
\NormalTok{H <{-}}\StringTok{ }\DecValTok{10000}

\CommentTok{\# Number of days between now and the forecast horizon}
\NormalTok{t <{-}}\StringTok{ }\DecValTok{5}

\CommentTok{\# Use set seed for testing purpose? Yes (1), No (0).}
\KeywordTok{source}\NormalTok{(}\DataTypeTok{file =} \KeywordTok{here}\NormalTok{(}\StringTok{"Functions"}\NormalTok{, }\StringTok{"seed.r"}\NormalTok{))}
\NormalTok{use\_set\_seed <{-}}\StringTok{ }\DecValTok{1}
\KeywordTok{seed}\NormalTok{(use\_set\_seed)}

\CommentTok{\# Store in \textquotesingle{}sim\_ret\_1\textquotesingle{} normally distributed IID shocks with mean \textquotesingle{}mu\_hat\textquotesingle{} and standard deviation \textquotesingle{}sd\_hat\textquotesingle{}}
\NormalTok{sim\_ret\_}\DecValTok{1}\NormalTok{ <{-}}\StringTok{ }\KeywordTok{matrix}\NormalTok{(}\KeywordTok{rnorm}\NormalTok{(t }\OperatorTok{*}\StringTok{ }\NormalTok{H, }\DataTypeTok{mean =}\NormalTok{ theta\_}\DecValTok{1}\NormalTok{[}\DecValTok{1}\NormalTok{], }\DataTypeTok{sd =}\NormalTok{ theta\_}\DecValTok{1}\NormalTok{[}\DecValTok{2}\NormalTok{]), }\DataTypeTok{nrow =}\NormalTok{ H, }\DataTypeTok{ncol =}\NormalTok{ t)}

\CommentTok{\# Histogram of the simulated returns}
\KeywordTok{hist}\NormalTok{(sim\_ret\_}\DecValTok{1}\NormalTok{[,}\DecValTok{5}\NormalTok{], }\DataTypeTok{nclass =} \KeywordTok{round}\NormalTok{(}\DecValTok{10} \OperatorTok{*}\StringTok{ }\KeywordTok{log}\NormalTok{(}\KeywordTok{length}\NormalTok{(sim\_ret\_}\DecValTok{1}\NormalTok{))),}
                \DataTypeTok{probability =} \OtherTok{TRUE}\NormalTok{,}
                \DataTypeTok{main =} \StringTok{"Histogram of 10 000 Simulated Returns"}\NormalTok{,}
                \DataTypeTok{xlab =} \StringTok{"Simulated Returns"}\NormalTok{)}
\end{Highlighting}
\end{Shaded}

\includegraphics{Project_Main_Notebook_Final_files/figure-latex/unnamed-chunk-7-1.pdf}

In order to obtain the underlying asset price one week from now, we
simply have to compute the exponential sum of the five daily normally
distributed IID shocks per trajectory and multiply by the latest
underlying asset price. The last things we need to modify are the
risk-free rate for the remaining life of the option and the remaining
days of the option. The strike price does not change over the option
life and the volatility is assumed to be constant.

\begin{Shaded}
\begin{Highlighting}[]
\CommentTok{\# Compute the price of the underlying asset one week from now}
\NormalTok{sim\_S\_}\DecValTok{1}\NormalTok{ <{-}}\StringTok{ }\NormalTok{S0 }\OperatorTok{*}\StringTok{ }\KeywordTok{exp}\NormalTok{(}\KeywordTok{rowSums}\NormalTok{(sim\_ret\_}\DecValTok{1}\NormalTok{))}

\CommentTok{\# Update the risk{-}free rate (360{-}day basis year)}
\NormalTok{rf\_m\_t <{-}}\StringTok{ }\KeywordTok{mapply}\NormalTok{(lin\_inter, (m }\OperatorTok{{-}}\StringTok{ }\NormalTok{t }\OperatorTok{/}\StringTok{ }\DecValTok{360}\NormalTok{))}

\CommentTok{\# Initialize a matrix to store call prices (10 000 rows (1 per simulation), 4 columns (1 per option))}
\NormalTok{sim\_price\_}\DecValTok{1}\NormalTok{ <{-}}\StringTok{ }\KeywordTok{matrix}\NormalTok{(}\OtherTok{NA}\NormalTok{, }\DataTypeTok{nrow =}\NormalTok{ H, }\DataTypeTok{ncol =} \DecValTok{4}\NormalTok{)}

\CommentTok{\# Loop through H simulations and price the options}
\ControlFlowTok{for}\NormalTok{ (i }\ControlFlowTok{in} \DecValTok{1}\OperatorTok{:}\NormalTok{H)\{}
\NormalTok{  sim\_price\_}\DecValTok{1}\NormalTok{[i,] <{-}}\StringTok{ }\KeywordTok{mapply}\NormalTok{(price\_BS, sim\_S\_}\DecValTok{1}\NormalTok{[i], K, rf\_m\_t, Vol, M }\OperatorTok{{-}}\StringTok{ }\NormalTok{t }\OperatorTok{/}\StringTok{ }\DecValTok{250}\NormalTok{, Type)}
\NormalTok{\}}

\CommentTok{\# Histogram of the simulated prices}
\KeywordTok{hist}\NormalTok{(sim\_S\_}\DecValTok{1}\NormalTok{, }\DataTypeTok{nclass =} \KeywordTok{round}\NormalTok{(}\DecValTok{10} \OperatorTok{*}\StringTok{ }\KeywordTok{log}\NormalTok{(}\KeywordTok{length}\NormalTok{(sim\_S\_}\DecValTok{1}\NormalTok{))),}
              \DataTypeTok{probability =} \OtherTok{TRUE}\NormalTok{, }\DataTypeTok{xlim =} \KeywordTok{c}\NormalTok{(}\DecValTok{1500}\NormalTok{,}\DecValTok{1900}\NormalTok{),}
              \DataTypeTok{main =} \StringTok{"Histogram of 10 000 Simulated Prices in 5 days"}\NormalTok{,}
              \DataTypeTok{xlab =} \StringTok{"Simulated Prices"}\NormalTok{)}

\CommentTok{\# Add a verticale line that represents the spot price}
\KeywordTok{abline}\NormalTok{(}\DataTypeTok{v   =}\NormalTok{ S0,}
       \DataTypeTok{lty =} \DecValTok{1}\NormalTok{,}
       \DataTypeTok{lwd =} \FloatTok{2.5}\NormalTok{,}
       \DataTypeTok{col =} \StringTok{"blue"}\NormalTok{)}
\end{Highlighting}
\end{Shaded}

\includegraphics{Project_Main_Notebook_Final_files/figure-latex/unnamed-chunk-8-1.pdf}

For each replication, we can compute the value of the portfolio of
options. Recall that we want to estimate the VaR and the ES of this
portfolio of options over the course of the following week. Therefore,
for each replication, we must compute a P\&L by discounting at the
risk-free rate the value of the portfolio of options one week from now
and subtracting the value of the portfolio observed today.

\begin{Shaded}
\begin{Highlighting}[]
\CommentTok{\# Compute the price of the portfolio for each replication and store it in \textquotesingle{}PF\_val\_1\textquotesingle{}}
\NormalTok{PF\_val\_}\DecValTok{1}\NormalTok{ <{-}}\StringTok{ }\KeywordTok{rowSums}\NormalTok{(sim\_price\_}\DecValTok{1} \OperatorTok{*}\StringTok{ }\NormalTok{book[,}\DecValTok{1}\NormalTok{])}

\CommentTok{\# Histogram of portfolio\textquotesingle{}s value}
\KeywordTok{hist}\NormalTok{(PF\_val\_}\DecValTok{1}\NormalTok{, }\DataTypeTok{nclass =} \KeywordTok{round}\NormalTok{(}\DecValTok{10} \OperatorTok{*}\StringTok{ }\KeywordTok{log}\NormalTok{(}\KeywordTok{length}\NormalTok{(PF\_val\_}\DecValTok{1}\NormalTok{))),}
               \DataTypeTok{probability =} \OtherTok{TRUE}\NormalTok{, }\DataTypeTok{xlim =} \KeywordTok{c}\NormalTok{(}\DecValTok{0}\NormalTok{,}\DecValTok{700}\NormalTok{),}
               \DataTypeTok{main =} \StringTok{"Histogram of 10 000 simulated portfolios value"}\NormalTok{,}
               \DataTypeTok{xlab =} \StringTok{"Simulated portfolios values"}\NormalTok{)}

\CommentTok{\# Add a vertical that represent last portfolio\textquotesingle{}s value}
\KeywordTok{abline}\NormalTok{(}\DataTypeTok{v   =}\NormalTok{ PF\_val,}
       \DataTypeTok{lty =} \DecValTok{1}\NormalTok{,}
       \DataTypeTok{lwd =} \FloatTok{2.5}\NormalTok{,}
       \DataTypeTok{col =} \StringTok{"blue"}\NormalTok{)}
\end{Highlighting}
\end{Shaded}

\includegraphics{Project_Main_Notebook_Final_files/figure-latex/unnamed-chunk-9-1.pdf}

Based on the histogram of the simulated portfolio values, we notice that
the porfolio of options as a function of underlying asset price at t = 0
is right skewed. Call option prices are therefore more sensitive to a
rise in the price of the underlying asset rather than a decline.

\begin{Shaded}
\begin{Highlighting}[]
\CommentTok{\# Compute the risk{-}free rate for a period of 5 days (360{-}day basis year)}
\NormalTok{rf\_t <{-}}\StringTok{ }\KeywordTok{lin\_inter}\NormalTok{(t }\OperatorTok{/}\StringTok{ }\DecValTok{360}\NormalTok{)}

\CommentTok{\# Compute the P\&L}
\NormalTok{PL\_}\DecValTok{1}\NormalTok{ <{-}}\StringTok{ }\NormalTok{PF\_val\_}\DecValTok{1} \OperatorTok{*}\StringTok{ }\KeywordTok{exp}\NormalTok{(}\OperatorTok{{-}}\NormalTok{(t }\OperatorTok{/}\StringTok{ }\DecValTok{360}\NormalTok{) }\OperatorTok{*}\StringTok{ }\NormalTok{rf\_t) }\OperatorTok{{-}}\StringTok{ }\NormalTok{PF\_val}
\end{Highlighting}
\end{Shaded}

The last step is to compute the VaR and the ES of the portfolio of
options P\&L distribution. In order to do so, we sort the P\&L values in
ascending order and identify the (1-alpha)-quartile of the empirical
distribution for the VaR, where alpha is the risk level (in our case,
0.95). The ES is simply the mean of the P\&L values smaller than the
VaR.

\begin{Shaded}
\begin{Highlighting}[]
\CommentTok{\# Set alpha to a desired significance level}
\NormalTok{alpha <{-}}\StringTok{ }\FloatTok{0.95}

\CommentTok{\# Compute the VaR and the ES of the P\&L distribution}
\NormalTok{VaR\_}\DecValTok{1}\NormalTok{ <{-}}\StringTok{ }\KeywordTok{sort}\NormalTok{(PL\_}\DecValTok{1}\NormalTok{)[(}\DecValTok{1} \OperatorTok{{-}}\StringTok{ }\NormalTok{alpha) }\OperatorTok{*}\StringTok{ }\NormalTok{H]}
\NormalTok{ES\_}\DecValTok{1}\NormalTok{  <{-}}\StringTok{ }\KeywordTok{mean}\NormalTok{(}\KeywordTok{sort}\NormalTok{(PL\_}\DecValTok{1}\NormalTok{)[}\DecValTok{1}\OperatorTok{:}\NormalTok{((}\DecValTok{1} \OperatorTok{{-}}\StringTok{ }\NormalTok{alpha) }\OperatorTok{*}\StringTok{ }\NormalTok{H)])}

\CommentTok{\# Plot an histogram}
\KeywordTok{hist}\NormalTok{(PL\_}\DecValTok{1}\NormalTok{, }\DataTypeTok{nclass =} \KeywordTok{round}\NormalTok{(}\DecValTok{10} \OperatorTok{*}\StringTok{ }\KeywordTok{log}\NormalTok{(}\KeywordTok{length}\NormalTok{(PL\_}\DecValTok{1}\NormalTok{))), }
           \DataTypeTok{probability =} \OtherTok{TRUE}\NormalTok{, }\DataTypeTok{xlim =} \KeywordTok{c}\NormalTok{(}\OperatorTok{{-}}\DecValTok{200}\NormalTok{,}\DecValTok{600}\NormalTok{),}
           \DataTypeTok{main =} \StringTok{"Histogram of 10 000 simulated PL\_1"}\NormalTok{)}

\CommentTok{\# Add a vertical line to show the VaR}
\KeywordTok{abline}\NormalTok{(}\DataTypeTok{v   =} \KeywordTok{quantile}\NormalTok{(PL\_}\DecValTok{1}\NormalTok{, }\DataTypeTok{probs =}\NormalTok{ (}\DecValTok{1} \OperatorTok{{-}}\StringTok{ }\NormalTok{alpha)),}
       \DataTypeTok{lty =} \DecValTok{1}\NormalTok{,}
       \DataTypeTok{lwd =} \FloatTok{2.5}\NormalTok{,}
       \DataTypeTok{col =} \StringTok{"red"}\NormalTok{)}
\end{Highlighting}
\end{Shaded}

\includegraphics{Project_Main_Notebook_Final_files/figure-latex/unnamed-chunk-11-1.pdf}

\begin{verbatim}
## The value at risk at alpha 0.95 is -122.24$.
\end{verbatim}

\begin{verbatim}
## The expected shortfall at alpha 0.95 is -134.80$.
\end{verbatim}

\hypertarget{analysis-portfolio-drivers}{%
\subsection{Analysis portfolio
drivers}\label{analysis-portfolio-drivers}}

This section explain what drive the portfolio price: What drive
portfolio value is the inputs of our models (BlackShcole formula).
Strike price and volatility are constants, risk-free rate is stable. So
distribution of the portfolio value can be explained by price of
underlying asset and days remaining until maturity.

Porfolio value is right skewed due to the relationship between calls
value and underlying price at t = 0.

Here we observe the value of the calls and the portfolio of the calls 5
days later. We isolated the change in portfolio value due to days until
maturity change only i.e.~without change in the underlying asset price,
volatility and risk-free rate.

\begin{Shaded}
\begin{Highlighting}[]
\NormalTok{day\_s =}\StringTok{ }\DecValTok{5}\OperatorTok{/}\DecValTok{250}
\NormalTok{book\_5day\_later      <{-}}\StringTok{ }\NormalTok{book[,}\DecValTok{1}\OperatorTok{:}\DecValTok{6}\NormalTok{]}
\NormalTok{book\_5day\_later[,}\DecValTok{4}\NormalTok{]  <{-}}\StringTok{ }\NormalTok{book[,}\DecValTok{4}\NormalTok{] }\OperatorTok{{-}}\StringTok{ }\NormalTok{day\_s}

\NormalTok{book\_5day\_later[,}\DecValTok{5}\NormalTok{] <{-}}\StringTok{ }\KeywordTok{mapply}\NormalTok{(price\_BS, S, K, rf, Vol, M }\OperatorTok{{-}}\StringTok{ }\NormalTok{day\_s, Type)}
\NormalTok{book\_5day\_later[,}\DecValTok{6}\NormalTok{] <{-}}\StringTok{ }\NormalTok{book\_5day\_later[,}\DecValTok{1}\NormalTok{]}\OperatorTok{*}\NormalTok{book\_5day\_later[,}\DecValTok{5}\NormalTok{]}

\NormalTok{PF\_val\_M\_choc <{-}}\StringTok{ }\KeywordTok{sum}\NormalTok{(book\_5day\_later[,}\DecValTok{6}\NormalTok{])}

\NormalTok{M\_choc <{-}}\StringTok{ }\NormalTok{PF\_val\_M\_choc }\OperatorTok{{-}}\StringTok{ }\NormalTok{PF\_val}
\end{Highlighting}
\end{Shaded}

\begin{verbatim}
## Calls value and maturity usually have positive relationship, and we can observe that 5 days in less until maturity lower
\end{verbatim}

\begin{verbatim}
## portfolio value by 8.06$.
\end{verbatim}

Summary of P\&L distribution and it's drivers

\begin{enumerate}
\def\labelenumi{\arabic{enumi}.}
\tightlist
\item
  Portfolio price and underlying price have positive relationship
\item
  P\&L is right skewed because of the relationship between Calls value
  and underlying asset price at t = 0
\item
  Portfolio value is more sensible to a positive change in uderlying
  asset price than a negative one
\item
  5 days in less until maturity make the portolio loose 8.06\$ in value
  (for a same underlying asset price)
\end{enumerate}

\hypertarget{part-v-risk-model-2-two-risk-drivers-and-gaussian-distribution}{%
\section{Part V: Risk Model 2: Two Risk Drivers and Gaussian
Distribution}\label{part-v-risk-model-2-two-risk-drivers-and-gaussian-distribution}}

The biggest assumption of the Risk Model 1 is that the volatility is
assumed constant between now and the following week. In practice,
volatility is not constant as evidenced by the time series of the VIX
index as a function of time. To account for this, we allow volatility to
vary in Risk Model 2. More specifically, we model the dynamics of the
underlying asset price and volatility with a multivariate normal
distribution that we calibrate using the S\&P500 and VIX log returns.

\begin{Shaded}
\begin{Highlighting}[]
\CommentTok{\# install.packages("MASS")}
\KeywordTok{library}\NormalTok{(}\StringTok{"MASS"}\NormalTok{)}

\CommentTok{\# Daily VIX log returns}
\NormalTok{vix\_return <{-}}\StringTok{ }\KeywordTok{diff}\NormalTok{(}\KeywordTok{log}\NormalTok{(vix))}

\CommentTok{\# Store underlying asset log returns and VIX log returns in \textquotesingle{}rets\textquotesingle{}}
\NormalTok{rets\_}\DecValTok{2}\NormalTok{     <{-}}\StringTok{ }\KeywordTok{matrix}\NormalTok{(}\OtherTok{NA}\NormalTok{, }\DataTypeTok{nrow =}\NormalTok{ n\_ret, }\DataTypeTok{ncol =} \DecValTok{2}\NormalTok{)}
\NormalTok{rets\_}\DecValTok{2}\NormalTok{[,}\DecValTok{1}\NormalTok{] <{-}}\StringTok{ }\NormalTok{log\_return}
\NormalTok{rets\_}\DecValTok{2}\NormalTok{[,}\DecValTok{2}\NormalTok{] <{-}}\StringTok{ }\NormalTok{vix\_return}

\CommentTok{\# Calibration}
\NormalTok{mu\_hat\_}\DecValTok{2}\NormalTok{ <{-}}\StringTok{ }\KeywordTok{colMeans}\NormalTok{(rets\_}\DecValTok{2}\NormalTok{)}
\NormalTok{sg\_hat\_}\DecValTok{2}\NormalTok{ <{-}}\StringTok{ }\NormalTok{((n\_ret }\OperatorTok{{-}}\StringTok{ }\DecValTok{1}\NormalTok{) }\OperatorTok{/}\StringTok{ }\NormalTok{n\_ret) }\OperatorTok{*}\StringTok{ }\KeywordTok{cov}\NormalTok{(rets\_}\DecValTok{2}\NormalTok{)}

\CommentTok{\# Store parameters in \textquotesingle{}theta\_2\textquotesingle{}}
\NormalTok{theta\_}\DecValTok{2}\NormalTok{ <{-}}\StringTok{ }\KeywordTok{list}\NormalTok{(}\DataTypeTok{mu =}\NormalTok{ mu\_hat\_}\DecValTok{2}\NormalTok{, }\DataTypeTok{sigma =}\NormalTok{ sg\_hat\_}\DecValTok{2}\NormalTok{)}
\end{Highlighting}
\end{Shaded}

We run a simulation of the underlying asset price and volatility index
value one week from now and re-price our portfolio of options in each
scenario.

\begin{Shaded}
\begin{Highlighting}[]
\CommentTok{\# Initialize the array \textquotesingle{}sim\_ret\_2\textquotesingle{}}
\NormalTok{sim\_ret\_}\DecValTok{2}\NormalTok{ <{-}}\StringTok{ }\KeywordTok{array}\NormalTok{(}\DataTypeTok{data =} \OtherTok{NA}\NormalTok{, }\DataTypeTok{dim =} \KeywordTok{c}\NormalTok{(H, }\DecValTok{2}\NormalTok{, t))}

\CommentTok{\# Set seed for generating pseudo{-}random numbers}
\KeywordTok{seed}\NormalTok{(use\_set\_seed)}

\CommentTok{\# Store in \textquotesingle{}sim\_ret\_2\textquotesingle{} daily shocks from a multivariate normal distribution with parameters \textquotesingle{}theta\_2\textquotesingle{}}
\ControlFlowTok{for}\NormalTok{ (i }\ControlFlowTok{in} \DecValTok{1}\OperatorTok{:}\NormalTok{t) \{}
\NormalTok{  sim\_ret\_}\DecValTok{2}\NormalTok{[,,i] <{-}}\StringTok{ }\KeywordTok{mvrnorm}\NormalTok{(}\DataTypeTok{n =}\NormalTok{ H, }\DataTypeTok{mu =}\NormalTok{ theta\_}\DecValTok{2}\OperatorTok{$}\NormalTok{mu, }\DataTypeTok{Sigma =}\NormalTok{ theta\_}\DecValTok{2}\OperatorTok{$}\NormalTok{sigma)}
\NormalTok{\}}

\CommentTok{\# Initialize two vectors that contain the underlying asset price and the VIX value one week from now}
\NormalTok{sim\_S\_}\DecValTok{2}\NormalTok{ <{-}}\StringTok{ }\KeywordTok{rep}\NormalTok{(}\OtherTok{NA}\NormalTok{, H)}
\NormalTok{sim\_vol\_}\DecValTok{2}\NormalTok{ <{-}}\StringTok{ }\KeywordTok{rep}\NormalTok{(}\OtherTok{NA}\NormalTok{, H)}

\CommentTok{\# Compute the underlying asset price and the VIX value one week from now}
\ControlFlowTok{for}\NormalTok{ (i }\ControlFlowTok{in} \DecValTok{1}\OperatorTok{:}\NormalTok{H) \{}
\NormalTok{  sim\_S\_}\DecValTok{2}\NormalTok{[i] <{-}}\StringTok{ }\NormalTok{S0 }\OperatorTok{*}\StringTok{ }\KeywordTok{exp}\NormalTok{(}\KeywordTok{sum}\NormalTok{(sim\_ret\_}\DecValTok{2}\NormalTok{[i,}\DecValTok{1}\NormalTok{,]))}
\NormalTok{  sim\_vol\_}\DecValTok{2}\NormalTok{[i] <{-}}\StringTok{ }\NormalTok{Vol0 }\OperatorTok{*}\StringTok{ }\KeywordTok{exp}\NormalTok{(}\KeywordTok{sum}\NormalTok{(sim\_ret\_}\DecValTok{2}\NormalTok{[i,}\DecValTok{2}\NormalTok{,]))}
\NormalTok{\}}

\CommentTok{\# Initialize a matrix to store call prices (H rows (1 per simulation), 4 columns (1 per option))}
\NormalTok{sim\_price\_}\DecValTok{2}\NormalTok{ <{-}}\StringTok{ }\KeywordTok{matrix}\NormalTok{(}\OtherTok{NA}\NormalTok{, }\DataTypeTok{nrow =}\NormalTok{ H, }\DataTypeTok{ncol =} \DecValTok{4}\NormalTok{)}

\CommentTok{\# Loop through H simulations and price each option}
\ControlFlowTok{for}\NormalTok{ (i }\ControlFlowTok{in} \DecValTok{1}\OperatorTok{:}\NormalTok{H)\{}
\NormalTok{  sim\_price\_}\DecValTok{2}\NormalTok{[i,] <{-}}\StringTok{ }\KeywordTok{mapply}\NormalTok{(price\_BS, sim\_S\_}\DecValTok{2}\NormalTok{[i], K, rf\_m\_t, sim\_vol\_}\DecValTok{2}\NormalTok{[i], M }\OperatorTok{{-}}\StringTok{ }\NormalTok{t }\OperatorTok{/}\StringTok{ }\DecValTok{250}\NormalTok{, Type)}
\NormalTok{\}}

\CommentTok{\# Compute the price of the portfolio for each replication and store it in \textquotesingle{}PF\_val\_2\textquotesingle{}}
\NormalTok{PF\_val\_}\DecValTok{2}\NormalTok{ <{-}}\StringTok{ }\KeywordTok{rowSums}\NormalTok{(sim\_price\_}\DecValTok{2} \OperatorTok{*}\StringTok{ }\NormalTok{book[,}\DecValTok{1}\NormalTok{])}
\end{Highlighting}
\end{Shaded}

Finally, we plot the P\&L distribution.

\begin{Shaded}
\begin{Highlighting}[]
\CommentTok{\# Compute the P\&L}
\NormalTok{PL\_}\DecValTok{2}\NormalTok{ <{-}}\StringTok{ }\NormalTok{PF\_val\_}\DecValTok{2} \OperatorTok{*}\StringTok{ }\KeywordTok{exp}\NormalTok{(}\OperatorTok{{-}}\NormalTok{(t }\OperatorTok{/}\StringTok{ }\DecValTok{360}\NormalTok{) }\OperatorTok{*}\StringTok{ }\NormalTok{rf\_t) }\OperatorTok{{-}}\StringTok{ }\NormalTok{PF\_val}

\CommentTok{\# Compute the VaR and the ES of the P\&L distribution}
\NormalTok{VaR\_}\DecValTok{2}\NormalTok{ <{-}}\StringTok{ }\KeywordTok{sort}\NormalTok{(PL\_}\DecValTok{2}\NormalTok{)[(}\DecValTok{1} \OperatorTok{{-}}\StringTok{ }\NormalTok{alpha) }\OperatorTok{*}\StringTok{ }\NormalTok{H]}
\NormalTok{ES\_}\DecValTok{2}\NormalTok{  <{-}}\StringTok{ }\KeywordTok{mean}\NormalTok{(}\KeywordTok{sort}\NormalTok{(PL\_}\DecValTok{2}\NormalTok{)[}\DecValTok{1}\OperatorTok{:}\NormalTok{((}\DecValTok{1} \OperatorTok{{-}}\StringTok{ }\NormalTok{alpha) }\OperatorTok{*}\StringTok{ }\NormalTok{H)])}

\CommentTok{\# Plot an histogram}
\KeywordTok{hist}\NormalTok{(PL\_}\DecValTok{2}\NormalTok{, }\DataTypeTok{nclass =} \KeywordTok{round}\NormalTok{(}\DecValTok{10} \OperatorTok{*}\StringTok{ }\KeywordTok{log}\NormalTok{(}\KeywordTok{length}\NormalTok{(PL\_}\DecValTok{2}\NormalTok{))), }
           \DataTypeTok{probability =} \OtherTok{TRUE}\NormalTok{, }\DataTypeTok{xlim =} \KeywordTok{c}\NormalTok{(}\OperatorTok{{-}}\DecValTok{200}\NormalTok{,}\DecValTok{600}\NormalTok{),}
           \DataTypeTok{main =} \StringTok{"Histogram of 10 000 simulated PL\_2"}\NormalTok{)}

\CommentTok{\# Add a vertical line to show the VaR}
\KeywordTok{abline}\NormalTok{(}\DataTypeTok{v   =} \KeywordTok{quantile}\NormalTok{(PL\_}\DecValTok{2}\NormalTok{, }\DataTypeTok{probs =}\NormalTok{ (}\DecValTok{1} \OperatorTok{{-}}\StringTok{ }\NormalTok{alpha)),}
       \DataTypeTok{lty =} \DecValTok{1}\NormalTok{,}
       \DataTypeTok{lwd =} \FloatTok{2.5}\NormalTok{,}
       \DataTypeTok{col =} \StringTok{"red"}\NormalTok{)}
\end{Highlighting}
\end{Shaded}

\includegraphics{Project_Main_Notebook_Final_files/figure-latex/unnamed-chunk-17-1.pdf}

\begin{verbatim}
## The value at risk at alpha 0.95 is -111.69$.
\end{verbatim}

\begin{verbatim}
## The expected shortfall at alpha 0.95 is -124.36$.
\end{verbatim}

Compared to Risk Model 1, Risk Model 2 has a higher (better) VaR and a
higher (better) ES for alpha = 0.95. This is not surprising knowing that
the correlation between the underlying asset log returns and the VIX log
returns is negative:

\begin{verbatim}
## The correlation between the S&P 500 and the VIX is -0.7537
\end{verbatim}

Since the value of a call option increases as the underlying asset and
volatility rise, and because these two variables are negatively
correlated, they have an offsetting effect on the value of a call
option. As a result, our portfolio of call options is less risky and has
a higher VaR.

\hypertarget{part-vi-risk-model-3-two-risk-drivers-and-copula-marginal-model-student-t-and-gaussian-copula}{%
\section{Part VI: Risk Model 3: Two Risk Drivers and Copula-Marginal
Model (Student-t and Gaussian
Copula)}\label{part-vi-risk-model-3-two-risk-drivers-and-copula-marginal-model-student-t-and-gaussian-copula}}

We add another layer on top of what we built up to now by introducing a
Gaussian copula in the mix. The copula models the dependence structure
of the two marginal distributions. In this case, we assume that the
marginals are student-t distributions with 10 degrees of freedom for the
underlying asset price and 5 degrees of freedom for the volatility.

The first step in implementing Risk Model 3 is to calibrate the two
marginal distributions using a similar methodology than previously
discussed.

\begin{Shaded}
\begin{Highlighting}[]
\CommentTok{\# install.packages("copula")}
\KeywordTok{library}\NormalTok{(}\StringTok{"copula"}\NormalTok{)}

\CommentTok{\# install.packages("fGarch")}
\KeywordTok{library}\NormalTok{(}\StringTok{"fGarch"}\NormalTok{)}
\end{Highlighting}
\end{Shaded}

\begin{verbatim}
## Loading required package: timeDate
\end{verbatim}

\begin{verbatim}
## Loading required package: timeSeries
\end{verbatim}

\begin{verbatim}
## Loading required package: fBasics
\end{verbatim}

\begin{Shaded}
\begin{Highlighting}[]
\CommentTok{\# Load the function}
\KeywordTok{source}\NormalTok{(}\DataTypeTok{file =} \KeywordTok{here}\NormalTok{(}\StringTok{"Functions"}\NormalTok{, }\StringTok{"nll\_student.r"}\NormalTok{)) }\CommentTok{\# Computes the negative log{-}likelihood function}

\CommentTok{\# Define an initial vector of parameters for the underlying asset log returns}
\NormalTok{theta\_}\DecValTok{0}\NormalTok{ <{-}}\StringTok{ }\KeywordTok{c}\NormalTok{(}\KeywordTok{mean}\NormalTok{(log\_return), }\KeywordTok{sd}\NormalTok{(log\_return), }\DecValTok{10}\NormalTok{)}

\CommentTok{\# Calibrate the student{-}t distribution on the underlying asset log returns}
\NormalTok{tmp <{-}}\StringTok{ }\KeywordTok{optim}\NormalTok{(}\DataTypeTok{par =}\NormalTok{ theta\_}\DecValTok{0}\NormalTok{,}
             \DataTypeTok{fn =}\NormalTok{ nll\_student,}
             \DataTypeTok{method =} \StringTok{"L{-}BFGS{-}B"}\NormalTok{,}
             \DataTypeTok{lower =} \KeywordTok{c}\NormalTok{(}\OperatorTok{{-}}\OtherTok{Inf}\NormalTok{, }\FloatTok{1e{-}5}\NormalTok{, }\DecValTok{10}\NormalTok{),}
             \DataTypeTok{x =}\NormalTok{ log\_return)}

\CommentTok{\# Store parameters in \textquotesingle{}theta\_S\_3\textquotesingle{}}
\NormalTok{theta\_S\_}\DecValTok{3}\NormalTok{ <{-}}\StringTok{ }\NormalTok{tmp}\OperatorTok{$}\NormalTok{par}

\CommentTok{\# Define an initial vector of parameters for the VIX log returns}
\NormalTok{theta\_}\DecValTok{0}\NormalTok{ <{-}}\StringTok{ }\KeywordTok{c}\NormalTok{(}\KeywordTok{mean}\NormalTok{(vix\_return), }\KeywordTok{sd}\NormalTok{(vix\_return), }\DecValTok{5}\NormalTok{)}

\CommentTok{\# Calibrate the student{-}t distribution on the VIX log returns}
\NormalTok{tmp <{-}}\StringTok{ }\KeywordTok{optim}\NormalTok{(}\DataTypeTok{par =}\NormalTok{ theta\_}\DecValTok{0}\NormalTok{,}
             \DataTypeTok{fn =}\NormalTok{ nll\_student,}
             \DataTypeTok{method =} \StringTok{"L{-}BFGS{-}B"}\NormalTok{,}
             \DataTypeTok{lower =} \KeywordTok{c}\NormalTok{(}\OperatorTok{{-}}\OtherTok{Inf}\NormalTok{, }\FloatTok{1e{-}5}\NormalTok{, }\DecValTok{5}\NormalTok{),}
             \DataTypeTok{x =}\NormalTok{ vix\_return)}

\CommentTok{\# Store parameters in \textquotesingle{}theta\_vol\_3\textquotesingle{}}
\NormalTok{theta\_vol\_}\DecValTok{3}\NormalTok{ <{-}}\StringTok{ }\NormalTok{tmp}\OperatorTok{$}\NormalTok{par}
\end{Highlighting}
\end{Shaded}

The second step in implementing Risk Model 3 is to use the probability
integral transformation theorem to obtain random variables that follow a
uniform distribution for both the underlying asset and the volatility
from their respective cumulative distribution function. We call these
variables `U\_1' and `U\_2' and use them to calibrate the Gaussian
copula.

\begin{Shaded}
\begin{Highlighting}[]
\CommentTok{\# Compute \textquotesingle{}U\_1\textquotesingle{} and \textquotesingle{}U\_2\textquotesingle{} and combine these two variables in \textquotesingle{}U\textquotesingle{}}
\NormalTok{U\_}\DecValTok{1}\NormalTok{ <{-}}\StringTok{ }\KeywordTok{pstd}\NormalTok{(log\_return, }\DataTypeTok{mean =}\NormalTok{ theta\_S\_}\DecValTok{3}\NormalTok{[}\DecValTok{1}\NormalTok{], }\DataTypeTok{sd =}\NormalTok{ theta\_S\_}\DecValTok{3}\NormalTok{[}\DecValTok{2}\NormalTok{], }\DataTypeTok{nu =}\NormalTok{ theta\_S\_}\DecValTok{3}\NormalTok{[}\DecValTok{3}\NormalTok{])}
\NormalTok{U\_}\DecValTok{2}\NormalTok{ <{-}}\StringTok{ }\KeywordTok{pstd}\NormalTok{(vix\_return, }\DataTypeTok{mean =}\NormalTok{ theta\_vol\_}\DecValTok{3}\NormalTok{[}\DecValTok{1}\NormalTok{], }\DataTypeTok{sd =}\NormalTok{ theta\_vol\_}\DecValTok{3}\NormalTok{[}\DecValTok{2}\NormalTok{], }\DataTypeTok{nu =}\NormalTok{ theta\_vol\_}\DecValTok{3}\NormalTok{[}\DecValTok{3}\NormalTok{])}
\NormalTok{U   <{-}}\StringTok{ }\KeywordTok{cbind}\NormalTok{(U\_}\DecValTok{1}\NormalTok{, U\_}\DecValTok{2}\NormalTok{)}

\CommentTok{\# Calibrate a Gaussian copula}
\NormalTok{C   <{-}}\StringTok{ }\KeywordTok{normalCopula}\NormalTok{(}\DataTypeTok{dim =} \DecValTok{2}\NormalTok{)}
\NormalTok{fit <{-}}\StringTok{ }\KeywordTok{fitCopula}\NormalTok{(C, }\DataTypeTok{data =}\NormalTok{ U, }\DataTypeTok{method =} \StringTok{"ml"}\NormalTok{)}

\CommentTok{\# Set seed for generating pseudo{-}random numbers}
\KeywordTok{seed}\NormalTok{(use\_set\_seed)}

\NormalTok{sim\_U       <{-}}\StringTok{ }\KeywordTok{rCopula}\NormalTok{(H }\OperatorTok{*}\StringTok{ }\NormalTok{t, fit}\OperatorTok{@}\NormalTok{copula)}
\NormalTok{sim\_log\_ret <{-}}\StringTok{ }\KeywordTok{qstd}\NormalTok{(sim\_U[,}\DecValTok{1}\NormalTok{], }\DataTypeTok{mean =}\NormalTok{ theta\_S\_}\DecValTok{3}\NormalTok{[}\DecValTok{1}\NormalTok{], }\DataTypeTok{sd =}\NormalTok{ theta\_S\_}\DecValTok{3}\NormalTok{[}\DecValTok{2}\NormalTok{], }\DataTypeTok{nu =}\NormalTok{ theta\_S\_}\DecValTok{3}\NormalTok{[}\DecValTok{3}\NormalTok{])}
\NormalTok{sim\_vix\_ret <{-}}\StringTok{ }\KeywordTok{qstd}\NormalTok{(sim\_U[,}\DecValTok{2}\NormalTok{], }\DataTypeTok{mean =}\NormalTok{ theta\_vol\_}\DecValTok{3}\NormalTok{[}\DecValTok{1}\NormalTok{], }\DataTypeTok{sd =}\NormalTok{ theta\_vol\_}\DecValTok{3}\NormalTok{[}\DecValTok{2}\NormalTok{], }\DataTypeTok{nu =}\NormalTok{ theta\_vol\_}\DecValTok{3}\NormalTok{[}\DecValTok{3}\NormalTok{])}
\end{Highlighting}
\end{Shaded}

We run a simulation of the underlying asset price and volatility index
value one week from now and re-price our portfolio of options in each
scenario.

\begin{Shaded}
\begin{Highlighting}[]
\CommentTok{\# Initialize the array \textquotesingle{}sim\_ret\_3\textquotesingle{}}
\NormalTok{sim\_ret\_}\DecValTok{3}\NormalTok{ <{-}}\StringTok{ }\KeywordTok{array}\NormalTok{(}\DataTypeTok{data =} \OtherTok{NA}\NormalTok{, }\DataTypeTok{dim =} \KeywordTok{c}\NormalTok{(H, }\DecValTok{2}\NormalTok{, t))}

\CommentTok{\# Store in \textquotesingle{}sim\_ret\_3\textquotesingle{} daily log returns for the underlying asset and the VIX}
\ControlFlowTok{for}\NormalTok{ (i }\ControlFlowTok{in} \DecValTok{1}\OperatorTok{:}\NormalTok{t) \{}
\NormalTok{  sim\_ret\_}\DecValTok{3}\NormalTok{[,,i] <{-}}\StringTok{ }\KeywordTok{c}\NormalTok{(sim\_log\_ret[(H }\OperatorTok{*}\StringTok{ }\NormalTok{(i }\OperatorTok{{-}}\StringTok{ }\DecValTok{1}\NormalTok{) }\OperatorTok{+}\StringTok{ }\DecValTok{1}\NormalTok{)}\OperatorTok{:}\NormalTok{(H }\OperatorTok{*}\StringTok{ }\NormalTok{i)], sim\_vix\_ret[(H }\OperatorTok{*}\StringTok{ }\NormalTok{(i }\OperatorTok{{-}}\StringTok{ }\DecValTok{1}\NormalTok{) }\OperatorTok{+}\StringTok{ }\DecValTok{1}\NormalTok{)}\OperatorTok{:}\NormalTok{(H }\OperatorTok{*}\StringTok{ }\NormalTok{i)])}
\NormalTok{\}}

\CommentTok{\# Initialize two vectors that contain the underlying asset price and the VIX value one week from now}
\NormalTok{sim\_S\_}\DecValTok{3}\NormalTok{ <{-}}\StringTok{ }\KeywordTok{rep}\NormalTok{(}\OtherTok{NA}\NormalTok{, H)}
\NormalTok{sim\_vol\_}\DecValTok{3}\NormalTok{ <{-}}\StringTok{ }\KeywordTok{rep}\NormalTok{(}\OtherTok{NA}\NormalTok{, H)}

\CommentTok{\# Compute the underlying asset price and the VIX value one week from now}
\ControlFlowTok{for}\NormalTok{ (i }\ControlFlowTok{in} \DecValTok{1}\OperatorTok{:}\NormalTok{H) \{}
\NormalTok{  sim\_S\_}\DecValTok{3}\NormalTok{[i]   <{-}}\StringTok{ }\NormalTok{S0 }\OperatorTok{*}\StringTok{ }\KeywordTok{exp}\NormalTok{(}\KeywordTok{sum}\NormalTok{(sim\_ret\_}\DecValTok{3}\NormalTok{[i,}\DecValTok{1}\NormalTok{,]))}
\NormalTok{  sim\_vol\_}\DecValTok{3}\NormalTok{[i] <{-}}\StringTok{ }\NormalTok{Vol0 }\OperatorTok{*}\StringTok{ }\KeywordTok{exp}\NormalTok{(}\KeywordTok{sum}\NormalTok{(sim\_ret\_}\DecValTok{3}\NormalTok{[i,}\DecValTok{2}\NormalTok{,]))}
\NormalTok{\}}

\CommentTok{\# Initialize a matrix to store call prices (H rows (1 per simulation), 4 columns (1 per option))}
\NormalTok{sim\_price\_}\DecValTok{3}\NormalTok{ <{-}}\StringTok{ }\KeywordTok{matrix}\NormalTok{(}\OtherTok{NA}\NormalTok{, }\DataTypeTok{nrow =}\NormalTok{ H, }\DataTypeTok{ncol =} \DecValTok{4}\NormalTok{)}

\CommentTok{\# Loop through H simulations and price each option}
\ControlFlowTok{for}\NormalTok{ (i }\ControlFlowTok{in} \DecValTok{1}\OperatorTok{:}\NormalTok{H)\{}
\NormalTok{  sim\_price\_}\DecValTok{3}\NormalTok{[i,] <{-}}\StringTok{ }\KeywordTok{mapply}\NormalTok{(price\_BS, sim\_S\_}\DecValTok{3}\NormalTok{[i], K, rf\_m\_t, sim\_vol\_}\DecValTok{3}\NormalTok{[i], M }\OperatorTok{{-}}\StringTok{ }\NormalTok{t }\OperatorTok{/}\StringTok{ }\DecValTok{250}\NormalTok{, Type)}
\NormalTok{\}}

\CommentTok{\# Compute the price of the portfolio for each replication and store it in \textquotesingle{}PF\_val\_3\textquotesingle{}}
\NormalTok{PF\_val\_}\DecValTok{3}\NormalTok{ <{-}}\StringTok{ }\KeywordTok{rowSums}\NormalTok{(sim\_price\_}\DecValTok{3} \OperatorTok{*}\StringTok{ }\NormalTok{book[,}\DecValTok{1}\NormalTok{])}
\end{Highlighting}
\end{Shaded}

We plot the P\&L distribution, as we did in previous parts.

\begin{Shaded}
\begin{Highlighting}[]
\CommentTok{\# Compute the P\&L}
\NormalTok{PL\_}\DecValTok{3}\NormalTok{ <{-}}\StringTok{ }\NormalTok{PF\_val\_}\DecValTok{3} \OperatorTok{*}\StringTok{ }\KeywordTok{exp}\NormalTok{(}\OperatorTok{{-}}\NormalTok{(t }\OperatorTok{/}\StringTok{ }\DecValTok{360}\NormalTok{) }\OperatorTok{*}\StringTok{ }\NormalTok{rf\_t) }\OperatorTok{{-}}\StringTok{ }\NormalTok{PF\_val}

\CommentTok{\# Compute the VaR and the ES of the P\&L distribution}
\NormalTok{VaR\_}\DecValTok{3}\NormalTok{ <{-}}\StringTok{ }\KeywordTok{sort}\NormalTok{(PL\_}\DecValTok{3}\NormalTok{)[(}\DecValTok{1} \OperatorTok{{-}}\StringTok{ }\NormalTok{alpha) }\OperatorTok{*}\StringTok{ }\NormalTok{H]}
\NormalTok{ES\_}\DecValTok{3}\NormalTok{  <{-}}\StringTok{ }\KeywordTok{mean}\NormalTok{(}\KeywordTok{sort}\NormalTok{(PL\_}\DecValTok{3}\NormalTok{)[}\DecValTok{1}\OperatorTok{:}\NormalTok{((}\DecValTok{1} \OperatorTok{{-}}\StringTok{ }\NormalTok{alpha) }\OperatorTok{*}\StringTok{ }\NormalTok{H)])}

\CommentTok{\# Plot an histogram}
\KeywordTok{hist}\NormalTok{(PL\_}\DecValTok{3}\NormalTok{, }\DataTypeTok{nclass =} \KeywordTok{round}\NormalTok{(}\DecValTok{10} \OperatorTok{*}\StringTok{ }\KeywordTok{log}\NormalTok{(}\KeywordTok{length}\NormalTok{(PL\_}\DecValTok{3}\NormalTok{))), }
           \DataTypeTok{probability =} \OtherTok{TRUE}\NormalTok{, }\DataTypeTok{xlim =} \KeywordTok{c}\NormalTok{(}\OperatorTok{{-}}\DecValTok{200}\NormalTok{,}\DecValTok{600}\NormalTok{),}
           \DataTypeTok{main =} \StringTok{"Histogram of 10 000 simulated PL\_3"}\NormalTok{)}

\CommentTok{\# Add a vertical line to show the VaR}
\KeywordTok{abline}\NormalTok{(}\DataTypeTok{v   =} \KeywordTok{quantile}\NormalTok{(PL\_}\DecValTok{3}\NormalTok{, }\DataTypeTok{probs =}\NormalTok{ (}\DecValTok{1} \OperatorTok{{-}}\StringTok{ }\NormalTok{alpha)),}
       \DataTypeTok{lty =} \DecValTok{1}\NormalTok{,}
       \DataTypeTok{lwd =} \FloatTok{2.5}\NormalTok{,}
       \DataTypeTok{col =} \StringTok{"red"}\NormalTok{)}
\end{Highlighting}
\end{Shaded}

\includegraphics{Project_Main_Notebook_Final_files/figure-latex/unnamed-chunk-23-1.pdf}

\begin{verbatim}
## The value at risk at alpha 0.95 is -98.67$.
\end{verbatim}

\begin{verbatim}
## The expected shortfall at alpha 0.95 is -112.77$.
\end{verbatim}

Compared to Risk Model 1 and Risk Model 2, Risk Model 3 has the smallest
VaR and the smallest ES for alpha = 0.95. As it was the case with Risk
Model 2, we account for the fact that the underlying asset price and the
volatility are negatively correlated. On top of that, we use the
Gaussian copula to model the dependence structure of the two marginal
distributions. The Gaussian copula underestimates the likelihood of
extreme negative events in the tail of the distributions.

\hypertarget{part-vii-risk-model-4-volatility-surface}{%
\section{Part VII: Risk Model 4: Volatility
Surface}\label{part-vii-risk-model-4-volatility-surface}}

In this section, we take a slightly different approach than with
previous risk models. Indeed, we fit a volatility surface on implied
volatilities of traded call and put options using the parametric model
described in the assignment.

\begin{Shaded}
\begin{Highlighting}[]
\CommentTok{\# Load the functions}
\KeywordTok{source}\NormalTok{(}\DataTypeTok{file =} \KeywordTok{here}\NormalTok{(}\StringTok{"Functions"}\NormalTok{, }\StringTok{"vol\_surface.R"}\NormalTok{))   }\CommentTok{\# Implied volatility of an option}
\KeywordTok{source}\NormalTok{(}\DataTypeTok{file =} \KeywordTok{here}\NormalTok{(}\StringTok{"Functions"}\NormalTok{, }\StringTok{"vol\_calibrate.r"}\NormalTok{)) }\CommentTok{\# Sum of absolute deviations of implied volalities}

\NormalTok{calls <{-}}\StringTok{ }\KeywordTok{as.matrix}\NormalTok{(Market}\OperatorTok{$}\NormalTok{calls)}
\NormalTok{puts  <{-}}\StringTok{ }\KeywordTok{as.matrix}\NormalTok{(Market}\OperatorTok{$}\NormalTok{puts)}
                   
\CommentTok{\# Count the number of traded options}
\NormalTok{nb\_opts <{-}}\StringTok{ }\KeywordTok{nrow}\NormalTok{(calls) }\OperatorTok{+}\StringTok{ }\KeywordTok{nrow}\NormalTok{(puts)}

\CommentTok{\# Build a matrix that contains the information relevant to traded call and put options}
\NormalTok{mkt\_vol <{-}}\StringTok{ }\KeywordTok{matrix}\NormalTok{(}\OtherTok{NA}\NormalTok{, }\DataTypeTok{nrow =}\NormalTok{ nb\_opts, }\DataTypeTok{ncol =} \DecValTok{4}\NormalTok{)}

\CommentTok{\# Assign names to columns}
\KeywordTok{colnames}\NormalTok{(mkt\_vol) <{-}}\StringTok{ }\KeywordTok{c}\NormalTok{(}\StringTok{"S"}\NormalTok{, }\StringTok{"K"}\NormalTok{, }\StringTok{"tau"}\NormalTok{, }\StringTok{"IV"}\NormalTok{)}

\CommentTok{\# Latest underlying asset price (spot price)}
\NormalTok{mkt\_vol[,}\DecValTok{1}\NormalTok{] <{-}}\StringTok{ }\KeywordTok{matrix}\NormalTok{(}\DataTypeTok{data =}\NormalTok{ S0, nb\_opts)}

\CommentTok{\# Strike price of options}
\NormalTok{mkt\_vol[,}\DecValTok{2}\NormalTok{] <{-}}\StringTok{ }\KeywordTok{c}\NormalTok{(calls[,}\DecValTok{1}\NormalTok{], puts[,}\DecValTok{1}\NormalTok{])}

\CommentTok{\# Time to expiry of options}
\NormalTok{mkt\_vol[,}\DecValTok{3}\NormalTok{] <{-}}\StringTok{ }\KeywordTok{c}\NormalTok{(calls[,}\DecValTok{2}\NormalTok{], puts[,}\DecValTok{2}\NormalTok{])}

\CommentTok{\# Implied volatility of options}
\NormalTok{mkt\_vol[,}\DecValTok{4}\NormalTok{] <{-}}\StringTok{ }\KeywordTok{c}\NormalTok{(calls[,}\DecValTok{3}\NormalTok{], puts[,}\DecValTok{3}\NormalTok{])}

\CommentTok{\# Set a vector of initial values of a1, a2, a3, and a4}
\NormalTok{x0 <{-}}\StringTok{ }\KeywordTok{c}\NormalTok{(}\FloatTok{0.2}\NormalTok{, }\DecValTok{1}\NormalTok{, }\DecValTok{1}\NormalTok{, }\FloatTok{0.1}\NormalTok{)}

\CommentTok{\# Calibrate the volatility surface on traded options}
\NormalTok{tmp <{-}}\StringTok{ }\KeywordTok{optim}\NormalTok{(}\DataTypeTok{par =}\NormalTok{ x0, }\DataTypeTok{fn =}\NormalTok{ vol\_calibrate)}

\CommentTok{\# Store parameters in \textquotesingle{}theta\_vol\textquotesingle{}}
\NormalTok{theta\_vol <{-}}\StringTok{ }\NormalTok{tmp}\OperatorTok{$}\NormalTok{par}
\end{Highlighting}
\end{Shaded}

We have a plot for the volatility surface. To reduce computation time,
we put it in comments.

\begin{Shaded}
\begin{Highlighting}[]
\CommentTok{\#install.packages("rgl")}
\CommentTok{\#library("rgl")}

\CommentTok{\# Generate a sequence of strike price and time to expiry}
\CommentTok{\#x1 <{-} S0 * seq(0.5, 1.5, (1.5 {-} 0.5) / 1000)}
\CommentTok{\#x2 <{-} seq(0.01, 2, (2 {-} 0.01) / 1000)}

\CommentTok{\# Generate al possible combinations of \textquotesingle{}x1\textquotesingle{} and \textquotesingle{}x2\textquotesingle{}}
\CommentTok{\#x3 <{-} expand.grid(x1,x2)}

\CommentTok{\# Compute the implied volatility for each combination}
\CommentTok{\#y  <{-} vol\_surface(S0, x3[,1], x3[,2], theta\_vol[1], theta\_vol[2], theta\_vol[3], theta\_vol[4])}

\CommentTok{\# Create a 3D{-}plot of the fitted volatility surface}
\CommentTok{\# plot3d(x3[,1], x3[,2], y)}
\end{Highlighting}
\end{Shaded}

Now that we have estimated the parameters that describe the shape of the
volatility surface observed at t = 0, we can compute the ATM implied
volatility given by the VIX. To estimate the ATM implied volatility in
five days, we assume that the ATM implied volatility difference stays
the same. Therefore, we use linear interpolation to project the implied
volatility in five days.

\begin{Shaded}
\begin{Highlighting}[]
\CommentTok{\# Compute the ATM implied volatility}
\NormalTok{vix\_1y\_atm\_IV <{-}}\StringTok{ }\NormalTok{theta\_vol[}\DecValTok{1}\NormalTok{] }\OperatorTok{+}\StringTok{ }\NormalTok{theta\_vol[}\DecValTok{4}\NormalTok{]}
\NormalTok{vix\_atm\_IV    <{-}}\StringTok{ }\KeywordTok{vol\_surface}\NormalTok{(S0, S0, M, theta\_vol[}\DecValTok{1}\NormalTok{], theta\_vol[}\DecValTok{2}\NormalTok{], theta\_vol[}\DecValTok{3}\NormalTok{], theta\_vol[}\DecValTok{4}\NormalTok{])}

\CommentTok{\# Compute the ATM implied volatility difference}
\NormalTok{vol\_shift     <{-}}\StringTok{ }\NormalTok{((vix\_1y\_atm\_IV }\OperatorTok{{-}}\StringTok{ }\NormalTok{vix\_atm\_IV) }\OperatorTok{/}\StringTok{ }\DecValTok{250}\NormalTok{) }\OperatorTok{*}\StringTok{ }\NormalTok{t}

\CommentTok{\# Set seed for generating pseudo{-}random numbers}
\KeywordTok{seed}\NormalTok{(use\_set\_seed)}

\CommentTok{\# Store in \textquotesingle{}sim\_ret\_4\textquotesingle{} normally distributed IID shocks }
\NormalTok{sim\_ret\_}\DecValTok{4}\NormalTok{ <{-}}\StringTok{ }\KeywordTok{matrix}\NormalTok{(}\KeywordTok{rnorm}\NormalTok{(t }\OperatorTok{*}\StringTok{ }\NormalTok{H, }\DataTypeTok{mean =} \KeywordTok{mean}\NormalTok{(log\_return), }\DataTypeTok{sd =} \KeywordTok{var}\NormalTok{(log\_return)}\OperatorTok{\^{}}\FloatTok{0.5}\NormalTok{), }\DataTypeTok{nrow =}\NormalTok{ H, }\DataTypeTok{ncol =}\NormalTok{ t)}

\CommentTok{\# Compute the underlying asset price one week from now}
\NormalTok{sim\_S\_}\DecValTok{4}\NormalTok{   <{-}}\StringTok{ }\NormalTok{S0 }\OperatorTok{*}\StringTok{ }\KeywordTok{exp}\NormalTok{(}\KeywordTok{rowSums}\NormalTok{(sim\_ret\_}\DecValTok{4}\NormalTok{))}

\CommentTok{\# Initialize a matrix to store call prices (H rows (1 per simulation), 4 columns (1 per option))}
\NormalTok{sim\_price\_}\DecValTok{4}\NormalTok{ <{-}}\StringTok{ }\KeywordTok{matrix}\NormalTok{(}\OtherTok{NA}\NormalTok{, }\DataTypeTok{nrow =}\NormalTok{ H, }\DataTypeTok{ncol =} \DecValTok{4}\NormalTok{)}

\CommentTok{\# Loop through H simulations and price each option}
\ControlFlowTok{for}\NormalTok{ (i }\ControlFlowTok{in} \DecValTok{1}\OperatorTok{:}\NormalTok{H)\{}
\NormalTok{  sim\_price\_}\DecValTok{4}\NormalTok{[i,] <{-}}\StringTok{ }\KeywordTok{mapply}\NormalTok{(price\_BS, sim\_S\_}\DecValTok{4}\NormalTok{[i], K, rf\_m\_t, Vol }\OperatorTok{+}\StringTok{ }\NormalTok{vol\_shift, M }\OperatorTok{{-}}\StringTok{ }\NormalTok{t }\OperatorTok{/}\StringTok{ }\DecValTok{250}\NormalTok{, Type)}
\NormalTok{\}}

\CommentTok{\# Compute the price of the portfolio for each replication and store it in \textquotesingle{}PF\_val\_4\textquotesingle{}}
\NormalTok{PF\_val\_}\DecValTok{4}\NormalTok{ <{-}}\StringTok{ }\KeywordTok{rowSums}\NormalTok{(sim\_price\_}\DecValTok{4} \OperatorTok{*}\StringTok{ }\NormalTok{book[,}\DecValTok{1}\NormalTok{])}
\end{Highlighting}
\end{Shaded}

We plot the P\&L distribution, as we did in previous parts.

\begin{Shaded}
\begin{Highlighting}[]
\CommentTok{\# Compute the P\&L}
\NormalTok{PL\_}\DecValTok{4}\NormalTok{ <{-}}\StringTok{ }\NormalTok{PF\_val\_}\DecValTok{4} \OperatorTok{*}\StringTok{ }\KeywordTok{exp}\NormalTok{(}\OperatorTok{{-}}\NormalTok{(t }\OperatorTok{/}\StringTok{ }\DecValTok{360}\NormalTok{) }\OperatorTok{*}\StringTok{ }\NormalTok{rf\_t) }\OperatorTok{{-}}\StringTok{ }\NormalTok{PF\_val}

\CommentTok{\# Compute the VaR and the ES of the P\&L distribution}
\NormalTok{VaR\_}\DecValTok{4}\NormalTok{ <{-}}\StringTok{ }\KeywordTok{sort}\NormalTok{(PL\_}\DecValTok{4}\NormalTok{)[(}\DecValTok{1} \OperatorTok{{-}}\StringTok{ }\NormalTok{alpha) }\OperatorTok{*}\StringTok{ }\NormalTok{H]}
\NormalTok{ES\_}\DecValTok{4}\NormalTok{  <{-}}\StringTok{ }\KeywordTok{mean}\NormalTok{(}\KeywordTok{sort}\NormalTok{(PL\_}\DecValTok{4}\NormalTok{)[}\DecValTok{1}\OperatorTok{:}\NormalTok{((}\DecValTok{1} \OperatorTok{{-}}\StringTok{ }\NormalTok{alpha) }\OperatorTok{*}\StringTok{ }\NormalTok{H)])}

\CommentTok{\# Plot an histogram}
\KeywordTok{hist}\NormalTok{(PL\_}\DecValTok{4}\NormalTok{, }\DataTypeTok{nclass =} \KeywordTok{round}\NormalTok{(}\DecValTok{10} \OperatorTok{*}\StringTok{ }\KeywordTok{log}\NormalTok{(}\KeywordTok{length}\NormalTok{(PL\_}\DecValTok{4}\NormalTok{))), }
           \DataTypeTok{probability =} \OtherTok{TRUE}\NormalTok{, }\DataTypeTok{xlim =} \KeywordTok{c}\NormalTok{(}\OperatorTok{{-}}\DecValTok{200}\NormalTok{,}\DecValTok{600}\NormalTok{),}
           \DataTypeTok{main =} \StringTok{"Histogram of 10 000 simulated PL\_4"}\NormalTok{)}

\CommentTok{\# Add a vertical line to show the VaR}
\KeywordTok{abline}\NormalTok{(}\DataTypeTok{v   =} \KeywordTok{quantile}\NormalTok{(PL\_}\DecValTok{4}\NormalTok{, }\DataTypeTok{probs =}\NormalTok{ (}\DecValTok{1} \OperatorTok{{-}}\StringTok{ }\NormalTok{alpha)),}
       \DataTypeTok{lty =} \DecValTok{1}\NormalTok{,}
       \DataTypeTok{lwd =} \FloatTok{2.5}\NormalTok{,}
       \DataTypeTok{col =} \StringTok{"red"}\NormalTok{)}
\end{Highlighting}
\end{Shaded}

\includegraphics{Project_Main_Notebook_Final_files/figure-latex/unnamed-chunk-28-1.pdf}

\begin{verbatim}
## The value at risk at alpha 0.95 is -122.51$.
\end{verbatim}

\begin{verbatim}
## The expected shortfall at alpha 0.95 is -135.02$.
\end{verbatim}

Compared to previous risk models, Risk Model 4 exhibits similar results
to the first risk model. This can be explained by the fact that by
isolating the effect of the shift of the volatility surface on option
prices, we did not take into account the negative correlation that
exists between the underlying asset price and the volatility. In this
case, we only generated stock returns and shifted the current implied
volatility by the one-year ATM implied volatility difference. The shift
reduced slightly the volatility of the options and therefore the VAR and
ES of P\&L is a bit lower than Risk Model 1.

\hypertarget{part-viii-risk-model-5-full-approach}{%
\section{Part VIII: Risk Model 5: Full
approach}\label{part-viii-risk-model-5-full-approach}}

In this section, instead of using the time series of our market data, we
use the residuals. We use a Garch(1,1) model for the log-returns of the
underlying asset and an AR(1) model for the volatiliy. We assume that
the residuals are normally distributed and use a gaussian copula to
model the dependence structure.

\begin{Shaded}
\begin{Highlighting}[]
\CommentTok{\#install.packages("rugarch")}
\KeywordTok{library}\NormalTok{(}\StringTok{"rugarch"}\NormalTok{)}
\end{Highlighting}
\end{Shaded}

\begin{verbatim}
## Loading required package: parallel
\end{verbatim}

\begin{verbatim}
## Registered S3 method overwritten by 'xts':
##   method     from
##   as.zoo.xts zoo
\end{verbatim}

\begin{verbatim}
## 
## Attaching package: 'rugarch'
\end{verbatim}

\begin{verbatim}
## The following object is masked from 'package:stats':
## 
##     sigma
\end{verbatim}

\begin{Shaded}
\begin{Highlighting}[]
\CommentTok{\# Residuals of the log{-}returns of the underlying using a Garch(1,1) with Normal innovations}
\NormalTok{spec   <{-}}\StringTok{ }\KeywordTok{ugarchspec}\NormalTok{(}\DataTypeTok{variance.model =} \KeywordTok{list}\NormalTok{(}\DataTypeTok{model =} \StringTok{"sGARCH"}\NormalTok{, }\DataTypeTok{garchOrder =} \KeywordTok{c}\NormalTok{(}\DecValTok{1}\NormalTok{,}\DecValTok{1}\NormalTok{)),}
                     \DataTypeTok{mean.model =} \KeywordTok{list}\NormalTok{(}\DataTypeTok{armaOrder =} \KeywordTok{c}\NormalTok{(}\DecValTok{0}\NormalTok{,}\DecValTok{0}\NormalTok{),}
                     \DataTypeTok{include.mean =} \OtherTok{FALSE}\NormalTok{),}
                     \DataTypeTok{distribution.model =} \StringTok{"norm"}\NormalTok{)}

\NormalTok{garch\_fit     <{-}}\StringTok{ }\KeywordTok{ugarchfit}\NormalTok{(}\DataTypeTok{spec =}\NormalTok{ spec, }\DataTypeTok{data =}\NormalTok{ log\_return)}
\NormalTok{Resid\_returns <{-}}\StringTok{ }\NormalTok{garch\_fit}\OperatorTok{@}\NormalTok{fit}\OperatorTok{$}\NormalTok{residuals}

\CommentTok{\# Residuals of the log{-}returns of the Vix using an AR(1) model }
\NormalTok{ar1\_vix   <{-}}\StringTok{ }\KeywordTok{arima}\NormalTok{(vix\_return, }\DataTypeTok{order =} \KeywordTok{c}\NormalTok{(}\DecValTok{1}\NormalTok{,}\DecValTok{0}\NormalTok{,}\DecValTok{0}\NormalTok{))}
\NormalTok{Resid\_vix <{-}}\StringTok{ }\NormalTok{ar1\_vix}\OperatorTok{$}\NormalTok{residuals}

\CommentTok{\# Fit normal marginals by MLE}
\NormalTok{fit1   <{-}}\StringTok{ }\KeywordTok{suppressWarnings}\NormalTok{(}\KeywordTok{fitdistr}\NormalTok{(}\DataTypeTok{x =}\NormalTok{ Resid\_returns,}
                         \DataTypeTok{densfun =}\NormalTok{ dnorm,}
                         \DataTypeTok{start =} \KeywordTok{list}\NormalTok{(}\DataTypeTok{mean =} \DecValTok{0}\NormalTok{, }\DataTypeTok{sd =} \DecValTok{1}\NormalTok{)))}
\NormalTok{theta1 <{-}}\StringTok{ }\NormalTok{fit1}\OperatorTok{$}\NormalTok{estimate}

\NormalTok{fit2   <{-}}\StringTok{ }\KeywordTok{suppressWarnings}\NormalTok{(}\KeywordTok{fitdistr}\NormalTok{(}\DataTypeTok{x =}\NormalTok{ Resid\_vix,}
                         \DataTypeTok{densfun =}\NormalTok{ dnorm,}
                         \DataTypeTok{start =} \KeywordTok{list}\NormalTok{(}\DataTypeTok{mean =} \DecValTok{0}\NormalTok{, }\DataTypeTok{sd =} \DecValTok{1}\NormalTok{)))}

\NormalTok{theta2 <{-}}\StringTok{ }\NormalTok{fit2}\OperatorTok{$}\NormalTok{estimate}

\CommentTok{\# Set seed for generating pseudo{-}random numbers}
\KeywordTok{seed}\NormalTok{(use\_set\_seed)}

\CommentTok{\# Compute \textquotesingle{}U\_1\textquotesingle{} and \textquotesingle{}U\_2\textquotesingle{} and combine these two variables in \textquotesingle{}U\textquotesingle{}}
\NormalTok{U1 <{-}}\StringTok{ }\KeywordTok{pnorm}\NormalTok{(Resid\_returns, }\DataTypeTok{mean =}\NormalTok{ theta1[}\DecValTok{1}\NormalTok{], }\DataTypeTok{sd =}\NormalTok{ theta1[}\DecValTok{2}\NormalTok{])}
\NormalTok{U2 <{-}}\StringTok{ }\KeywordTok{pnorm}\NormalTok{(Resid\_vix, }\DataTypeTok{mean =}\NormalTok{ theta2[}\DecValTok{1}\NormalTok{], }\DataTypeTok{sd =}\NormalTok{ theta2[}\DecValTok{2}\NormalTok{])}
\NormalTok{U  <{-}}\StringTok{ }\KeywordTok{cbind}\NormalTok{(U1, U2)}

\CommentTok{\# U1 <{-} pnorm(log\_return, mean = mean(log\_return), sd = var(log\_return)\^{}0.5)}
\CommentTok{\# U2 <{-} pnorm(vix\_return, mean = mean(vix\_return), sd = var(vix\_return)\^{}0.5)}
\CommentTok{\# U  <{-} cbind(U1, U2)}

\CommentTok{\# Calibrate a Gaussian copula}
\NormalTok{C   <{-}}\StringTok{ }\KeywordTok{normalCopula}\NormalTok{(}\DataTypeTok{dim =} \DecValTok{2}\NormalTok{)}
\NormalTok{fit <{-}}\StringTok{ }\KeywordTok{fitCopula}\NormalTok{(C, }\DataTypeTok{data =}\NormalTok{ U, }\DataTypeTok{method =} \StringTok{"ml"}\NormalTok{)}

\NormalTok{sim\_U             <{-}}\StringTok{ }\KeywordTok{rCopula}\NormalTok{(H }\OperatorTok{*}\StringTok{ }\NormalTok{t, fit}\OperatorTok{@}\NormalTok{copula)}
\NormalTok{sim\_Resid\_returns <{-}}\StringTok{ }\KeywordTok{qnorm}\NormalTok{(sim\_U[,}\DecValTok{1}\NormalTok{], }\DataTypeTok{mean =}\NormalTok{ theta1[}\DecValTok{1}\NormalTok{], }\DataTypeTok{sd =}\NormalTok{ theta1[}\DecValTok{2}\NormalTok{])}
\NormalTok{sim\_Resid\_vix     <{-}}\StringTok{ }\KeywordTok{qnorm}\NormalTok{(sim\_U[,}\DecValTok{2}\NormalTok{], }\DataTypeTok{mean =}\NormalTok{ theta2[}\DecValTok{1}\NormalTok{], }\DataTypeTok{sd =}\NormalTok{ theta2[}\DecValTok{2}\NormalTok{])}

\CommentTok{\# Initialize the array \textquotesingle{}sim\_ret\_5\textquotesingle{}}
\NormalTok{sim\_ret\_}\DecValTok{5}\NormalTok{ <{-}}\StringTok{ }\KeywordTok{array}\NormalTok{(}\DataTypeTok{data =} \OtherTok{NA}\NormalTok{, }\DataTypeTok{dim =} \KeywordTok{c}\NormalTok{(H, }\DecValTok{2}\NormalTok{, t))}

\CommentTok{\# Store in \textquotesingle{}sim\_ret\_5\textquotesingle{} daily residuals of log returns for the underlying asset and the VIX}
\ControlFlowTok{for}\NormalTok{ (i }\ControlFlowTok{in} \DecValTok{1}\OperatorTok{:}\NormalTok{t) \{}
\NormalTok{  sim\_ret\_}\DecValTok{5}\NormalTok{[,,i] <{-}}\StringTok{ }\KeywordTok{c}\NormalTok{(sim\_Resid\_returns[(H }\OperatorTok{*}\StringTok{ }\NormalTok{(i }\OperatorTok{{-}}\StringTok{ }\DecValTok{1}\NormalTok{) }\OperatorTok{+}\StringTok{ }\DecValTok{1}\NormalTok{)}\OperatorTok{:}\NormalTok{(H }\OperatorTok{*}\StringTok{ }\NormalTok{i)], sim\_Resid\_vix[(H }\OperatorTok{*}\StringTok{ }\NormalTok{(i }\OperatorTok{{-}}\StringTok{ }\DecValTok{1}\NormalTok{) }\OperatorTok{+}\StringTok{ }\DecValTok{1}\NormalTok{)}\OperatorTok{:}\NormalTok{(H }\OperatorTok{*}\StringTok{ }\NormalTok{i)])}
\NormalTok{\}}

\CommentTok{\# Initialize two vectors that contain the residuals of the underlying asset price and the VIX value one week from now}
\NormalTok{sim\_S\_}\DecValTok{5}\NormalTok{   <{-}}\StringTok{ }\KeywordTok{rep}\NormalTok{(}\OtherTok{NA}\NormalTok{, H)}
\NormalTok{sim\_vol\_}\DecValTok{5}\NormalTok{ <{-}}\StringTok{ }\KeywordTok{rep}\NormalTok{(}\OtherTok{NA}\NormalTok{, H)}

\CommentTok{\# Compute the underlying asset price and the VIX value one week from now}
\ControlFlowTok{for}\NormalTok{ (i }\ControlFlowTok{in} \DecValTok{1}\OperatorTok{:}\NormalTok{H) \{}
\NormalTok{  sim\_S\_}\DecValTok{5}\NormalTok{[i]   <{-}}\StringTok{ }\NormalTok{S0 }\OperatorTok{*}\StringTok{ }\KeywordTok{exp}\NormalTok{(}\KeywordTok{sum}\NormalTok{(sim\_ret\_}\DecValTok{5}\NormalTok{[i,}\DecValTok{1}\NormalTok{,]))}
\NormalTok{  sim\_vol\_}\DecValTok{5}\NormalTok{[i] <{-}}\StringTok{ }\NormalTok{Vol0 }\OperatorTok{*}\StringTok{ }\KeywordTok{exp}\NormalTok{(}\KeywordTok{sum}\NormalTok{(sim\_ret\_}\DecValTok{5}\NormalTok{[i,}\DecValTok{2}\NormalTok{,]))}
\NormalTok{\}}

\CommentTok{\# Initialize a matrix to store call prices (H rows (1 per simulation), 4 columns (1 per option))}
\NormalTok{sim\_price\_}\DecValTok{5}\NormalTok{ <{-}}\StringTok{ }\KeywordTok{matrix}\NormalTok{(}\OtherTok{NA}\NormalTok{, }\DataTypeTok{nrow =}\NormalTok{ H, }\DataTypeTok{ncol =} \DecValTok{4}\NormalTok{)}

\CommentTok{\# Loop through H simulations and price each option}
\ControlFlowTok{for}\NormalTok{ (i }\ControlFlowTok{in} \DecValTok{1}\OperatorTok{:}\NormalTok{H)\{}
\NormalTok{  sim\_price\_}\DecValTok{5}\NormalTok{[i,] <{-}}\StringTok{ }\KeywordTok{mapply}\NormalTok{(price\_BS, sim\_S\_}\DecValTok{5}\NormalTok{[i], K, rf\_m\_t, sim\_vol\_}\DecValTok{5}\NormalTok{[i], M }\OperatorTok{{-}}\StringTok{ }\NormalTok{t }\OperatorTok{/}\StringTok{ }\DecValTok{250}\NormalTok{, Type)}
\NormalTok{\}}
\end{Highlighting}
\end{Shaded}

We plot the P\&L distribution, as we did in previous parts.

\begin{Shaded}
\begin{Highlighting}[]
\CommentTok{\# Compute the price of the portfolio for each replication and store it in \textquotesingle{}PF\_val\_5\textquotesingle{}}
\NormalTok{PF\_val\_}\DecValTok{5}\NormalTok{ <{-}}\StringTok{ }\KeywordTok{rowSums}\NormalTok{(sim\_price\_}\DecValTok{5} \OperatorTok{*}\StringTok{ }\NormalTok{book[,}\DecValTok{1}\NormalTok{])}

\CommentTok{\# Compute the P\&L}
\NormalTok{PL\_}\DecValTok{5}\NormalTok{ <{-}}\StringTok{ }\NormalTok{PF\_val\_}\DecValTok{5} \OperatorTok{*}\StringTok{ }\KeywordTok{exp}\NormalTok{(}\OperatorTok{{-}}\NormalTok{(t }\OperatorTok{/}\StringTok{ }\DecValTok{360}\NormalTok{) }\OperatorTok{*}\StringTok{ }\NormalTok{rf\_t) }\OperatorTok{{-}}\StringTok{ }\NormalTok{PF\_val}

\CommentTok{\# Compute the VaR and the ES of the P\&L distribution}
\NormalTok{VaR\_}\DecValTok{5}\NormalTok{ <{-}}\StringTok{ }\KeywordTok{sort}\NormalTok{(PL\_}\DecValTok{5}\NormalTok{)[(}\DecValTok{1} \OperatorTok{{-}}\StringTok{ }\NormalTok{alpha) }\OperatorTok{*}\StringTok{ }\NormalTok{H]}
\NormalTok{ES\_}\DecValTok{5}\NormalTok{  <{-}}\StringTok{ }\KeywordTok{mean}\NormalTok{(}\KeywordTok{sort}\NormalTok{(PL\_}\DecValTok{5}\NormalTok{)[}\DecValTok{1}\OperatorTok{:}\NormalTok{((}\DecValTok{1} \OperatorTok{{-}}\StringTok{ }\NormalTok{alpha) }\OperatorTok{*}\StringTok{ }\NormalTok{H)])}

\CommentTok{\# Plot an histogram}
\KeywordTok{hist}\NormalTok{(PL\_}\DecValTok{5}\NormalTok{, }\DataTypeTok{nclass =} \KeywordTok{round}\NormalTok{(}\DecValTok{10} \OperatorTok{*}\StringTok{ }\KeywordTok{log}\NormalTok{(}\KeywordTok{length}\NormalTok{(PL\_}\DecValTok{5}\NormalTok{))), }
           \DataTypeTok{probability =} \OtherTok{TRUE}\NormalTok{, }\DataTypeTok{xlim =} \KeywordTok{c}\NormalTok{(}\OperatorTok{{-}}\DecValTok{200}\NormalTok{,}\DecValTok{600}\NormalTok{),}
           \DataTypeTok{main =} \StringTok{"Histogram of 10 000 simulated PL\_5"}\NormalTok{)}

\CommentTok{\# Add a vertical line to show the VaR}
\KeywordTok{abline}\NormalTok{(}\DataTypeTok{v   =} \KeywordTok{quantile}\NormalTok{(PL\_}\DecValTok{5}\NormalTok{, }\DataTypeTok{probs =}\NormalTok{ (}\DecValTok{1} \OperatorTok{{-}}\StringTok{ }\NormalTok{alpha)),}
       \DataTypeTok{lty =} \DecValTok{1}\NormalTok{,}
       \DataTypeTok{lwd =} \FloatTok{2.5}\NormalTok{,}
       \DataTypeTok{col =} \StringTok{"red"}\NormalTok{)}
\end{Highlighting}
\end{Shaded}

\includegraphics{Project_Main_Notebook_Final_files/figure-latex/unnamed-chunk-31-1.pdf}

\begin{verbatim}
## The value at risk at alpha 0.95 is -112.03$.
\end{verbatim}

\begin{verbatim}
## The expected shortfall at alpha 0.95 is -123.96$.
\end{verbatim}

The VAR and ES are similar to the second model. We also tried this model
versus using regular vix returns and underlying asset returns and end up
with similar results. This means that the Garch(1,1) and AR(1) models
doesn't really improve the results. Using the principle of Occam's
razor, between two model giving similar results, using the simpler model
is better.

\hypertarget{part-ix-results-and-conclusion}{%
\section{Part IX: Results and
Conclusion}\label{part-ix-results-and-conclusion}}

Here is a table summarizing the results of each model.

\begin{verbatim}
## Portfolio value 156.98$.
\end{verbatim}

\begin{verbatim}
## 
## Portfolio detail:
\end{verbatim}

\begin{verbatim}
##      Quantity Option Type Strike Price Maturity Option price
## [1,]    1.00     1.00     1600.00         0.08    87.57     
## [2,]    1.00     1.00     1650.00         0.08    47.72     
## [3,]    1.00     1.00     1750.00         0.16    15.30     
## [4,]    1.00     1.00     1800.00         0.16     6.38     
##      Position Value
## [1,]   87.57       
## [2,]   47.72       
## [3,]   15.30       
## [4,]    6.38
\end{verbatim}

\begin{verbatim}
## 
## Portfolio 5-days risk:
\end{verbatim}

\begin{verbatim}
##                    M1: 1 risk/Gauss M2: 2 risks/Gauss M3: 2 risk/Copula
## Value At Risk      -122.239         -111.686           -98.673         
## Expected Shortfall -134.800         -124.359          -112.774         
##                    M4: Vol Surface M5: Full Approach
## Value At Risk      -122.515        -112.025         
## Expected Shortfall -135.022        -123.956
\end{verbatim}


\end{document}
