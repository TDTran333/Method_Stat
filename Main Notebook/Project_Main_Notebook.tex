\documentclass[]{article}
\usepackage{lmodern}
\usepackage{amssymb,amsmath}
\usepackage{ifxetex,ifluatex}
\usepackage{fixltx2e} % provides \textsubscript
\ifnum 0\ifxetex 1\fi\ifluatex 1\fi=0 % if pdftex
  \usepackage[T1]{fontenc}
  \usepackage[utf8]{inputenc}
\else % if luatex or xelatex
  \ifxetex
    \usepackage{mathspec}
  \else
    \usepackage{fontspec}
  \fi
  \defaultfontfeatures{Ligatures=TeX,Scale=MatchLowercase}
\fi
% use upquote if available, for straight quotes in verbatim environments
\IfFileExists{upquote.sty}{\usepackage{upquote}}{}
% use microtype if available
\IfFileExists{microtype.sty}{%
\usepackage{microtype}
\UseMicrotypeSet[protrusion]{basicmath} % disable protrusion for tt fonts
}{}
\usepackage[margin=1in]{geometry}
\usepackage{hyperref}
\hypersetup{unicode=true,
            pdftitle={Statistical Methods in Financial Engineering - Risk Management Project},
            pdfauthor={Chloe Morin-Leclerc, Felix-Antoine Groulx, Denis Genest, Thien Duy Tran},
            pdfborder={0 0 0},
            breaklinks=true}
\urlstyle{same}  % don't use monospace font for urls
\usepackage{color}
\usepackage{fancyvrb}
\newcommand{\VerbBar}{|}
\newcommand{\VERB}{\Verb[commandchars=\\\{\}]}
\DefineVerbatimEnvironment{Highlighting}{Verbatim}{commandchars=\\\{\}}
% Add ',fontsize=\small' for more characters per line
\usepackage{framed}
\definecolor{shadecolor}{RGB}{248,248,248}
\newenvironment{Shaded}{\begin{snugshade}}{\end{snugshade}}
\newcommand{\AlertTok}[1]{\textcolor[rgb]{0.94,0.16,0.16}{#1}}
\newcommand{\AnnotationTok}[1]{\textcolor[rgb]{0.56,0.35,0.01}{\textbf{\textit{#1}}}}
\newcommand{\AttributeTok}[1]{\textcolor[rgb]{0.77,0.63,0.00}{#1}}
\newcommand{\BaseNTok}[1]{\textcolor[rgb]{0.00,0.00,0.81}{#1}}
\newcommand{\BuiltInTok}[1]{#1}
\newcommand{\CharTok}[1]{\textcolor[rgb]{0.31,0.60,0.02}{#1}}
\newcommand{\CommentTok}[1]{\textcolor[rgb]{0.56,0.35,0.01}{\textit{#1}}}
\newcommand{\CommentVarTok}[1]{\textcolor[rgb]{0.56,0.35,0.01}{\textbf{\textit{#1}}}}
\newcommand{\ConstantTok}[1]{\textcolor[rgb]{0.00,0.00,0.00}{#1}}
\newcommand{\ControlFlowTok}[1]{\textcolor[rgb]{0.13,0.29,0.53}{\textbf{#1}}}
\newcommand{\DataTypeTok}[1]{\textcolor[rgb]{0.13,0.29,0.53}{#1}}
\newcommand{\DecValTok}[1]{\textcolor[rgb]{0.00,0.00,0.81}{#1}}
\newcommand{\DocumentationTok}[1]{\textcolor[rgb]{0.56,0.35,0.01}{\textbf{\textit{#1}}}}
\newcommand{\ErrorTok}[1]{\textcolor[rgb]{0.64,0.00,0.00}{\textbf{#1}}}
\newcommand{\ExtensionTok}[1]{#1}
\newcommand{\FloatTok}[1]{\textcolor[rgb]{0.00,0.00,0.81}{#1}}
\newcommand{\FunctionTok}[1]{\textcolor[rgb]{0.00,0.00,0.00}{#1}}
\newcommand{\ImportTok}[1]{#1}
\newcommand{\InformationTok}[1]{\textcolor[rgb]{0.56,0.35,0.01}{\textbf{\textit{#1}}}}
\newcommand{\KeywordTok}[1]{\textcolor[rgb]{0.13,0.29,0.53}{\textbf{#1}}}
\newcommand{\NormalTok}[1]{#1}
\newcommand{\OperatorTok}[1]{\textcolor[rgb]{0.81,0.36,0.00}{\textbf{#1}}}
\newcommand{\OtherTok}[1]{\textcolor[rgb]{0.56,0.35,0.01}{#1}}
\newcommand{\PreprocessorTok}[1]{\textcolor[rgb]{0.56,0.35,0.01}{\textit{#1}}}
\newcommand{\RegionMarkerTok}[1]{#1}
\newcommand{\SpecialCharTok}[1]{\textcolor[rgb]{0.00,0.00,0.00}{#1}}
\newcommand{\SpecialStringTok}[1]{\textcolor[rgb]{0.31,0.60,0.02}{#1}}
\newcommand{\StringTok}[1]{\textcolor[rgb]{0.31,0.60,0.02}{#1}}
\newcommand{\VariableTok}[1]{\textcolor[rgb]{0.00,0.00,0.00}{#1}}
\newcommand{\VerbatimStringTok}[1]{\textcolor[rgb]{0.31,0.60,0.02}{#1}}
\newcommand{\WarningTok}[1]{\textcolor[rgb]{0.56,0.35,0.01}{\textbf{\textit{#1}}}}
\usepackage{graphicx,grffile}
\makeatletter
\def\maxwidth{\ifdim\Gin@nat@width>\linewidth\linewidth\else\Gin@nat@width\fi}
\def\maxheight{\ifdim\Gin@nat@height>\textheight\textheight\else\Gin@nat@height\fi}
\makeatother
% Scale images if necessary, so that they will not overflow the page
% margins by default, and it is still possible to overwrite the defaults
% using explicit options in \includegraphics[width, height, ...]{}
\setkeys{Gin}{width=\maxwidth,height=\maxheight,keepaspectratio}
\IfFileExists{parskip.sty}{%
\usepackage{parskip}
}{% else
\setlength{\parindent}{0pt}
\setlength{\parskip}{6pt plus 2pt minus 1pt}
}
\setlength{\emergencystretch}{3em}  % prevent overfull lines
\providecommand{\tightlist}{%
  \setlength{\itemsep}{0pt}\setlength{\parskip}{0pt}}
\setcounter{secnumdepth}{0}
% Redefines (sub)paragraphs to behave more like sections
\ifx\paragraph\undefined\else
\let\oldparagraph\paragraph
\renewcommand{\paragraph}[1]{\oldparagraph{#1}\mbox{}}
\fi
\ifx\subparagraph\undefined\else
\let\oldsubparagraph\subparagraph
\renewcommand{\subparagraph}[1]{\oldsubparagraph{#1}\mbox{}}
\fi

%%% Use protect on footnotes to avoid problems with footnotes in titles
\let\rmarkdownfootnote\footnote%
\def\footnote{\protect\rmarkdownfootnote}

%%% Change title format to be more compact
\usepackage{titling}

% Create subtitle command for use in maketitle
\providecommand{\subtitle}[1]{
  \posttitle{
    \begin{center}\large#1\end{center}
    }
}

\setlength{\droptitle}{-2em}

  \title{Statistical Methods in Financial Engineering - Risk Management Project}
    \pretitle{\vspace{\droptitle}\centering\huge}
  \posttitle{\par}
    \author{Chloe Morin-Leclerc, Felix-Antoine Groulx, Denis Genest, Thien Duy Tran}
    \preauthor{\centering\large\emph}
  \postauthor{\par}
      \predate{\centering\large\emph}
  \postdate{\par}
    \date{December 12, 2019}


\begin{document}
\maketitle

\hypertarget{part-i-project-guidelines}{%
\section{Part I: Project Guidelines}\label{part-i-project-guidelines}}

\hypertarget{context}{%
\subsection{Context}\label{context}}

You work as a quantitative analyst for a large investment bank. You and
your team are responsible for challenging the models used by traders and
risk managers. You work with R and love reproducible research. All your
files are written in Rmarkdown or R notebook.

You can watch the introductory video on Rmarkdown to help you build
properly the R file. Moreover, you use GitHub with your team. You'll
build a dedicated project for the tasks below and use RStudio with
GitHub extensively. You also use Loom to share your findings with other
teams in the investment bank.

\hypertarget{learning-objectives}{%
\subsection{Learning Objectives}\label{learning-objectives}}

\begin{enumerate}
\def\labelenumi{\arabic{enumi}.}
\tightlist
\item
  Content (scientific rigor, concepts, creativity).\\
\item
  Choose the right tools.\\
\item
  Implement the steps correctly.\\
\item
  Come up with innovative solutions.\\
\end{enumerate}

\hypertarget{form-coding-collaboration-and-presentation}{%
\subsection{Form: Coding, Collaboration and
Presentation}\label{form-coding-collaboration-and-presentation}}

\begin{enumerate}
\def\labelenumi{\arabic{enumi}.}
\tightlist
\item
  Build RStudio project with proper folder structure and
  Rmarkdown/nootebook file to reproduce your results.\\
\item
  Program with state-of-the-art coding standards.\\
\item
  Use GitHub repository for collaborative research.\\
\item
  Use Loom video for presenting your results.
\end{enumerate}

\hypertarget{objective}{%
\subsection{Objective}\label{objective}}

The objective of this project is to implement the risk management
framework used for estimating the risk of a book of European call
options by taking into account risk drivers such as the underlying asset
and the implied volatility of the options.

\hypertarget{part-ii-data}{%
\section{Part II: Data}\label{part-ii-data}}

\hypertarget{loading-the-data}{%
\subsection{Loading the Data}\label{loading-the-data}}

The first step is to load the database `Market'. `Market' is a list of 5
elements: S\&P500 index prices, VIX values, the term structure of
interest rates, and traded call and put options information. To make
sure this code can run on any platform, we use the library `here'.

\begin{Shaded}
\begin{Highlighting}[]
\CommentTok{# install.packages("here")}
\KeywordTok{library}\NormalTok{(}\StringTok{"here"}\NormalTok{)}
\end{Highlighting}
\end{Shaded}

\begin{verbatim}
## here() starts at C:/Users/TTD/Documents/GitHub/Method_Stat
\end{verbatim}

\begin{Shaded}
\begin{Highlighting}[]
\CommentTok{# Load the data}
\KeywordTok{load}\NormalTok{(}\DataTypeTok{file =} \KeywordTok{here}\NormalTok{(}\StringTok{"Data"}\NormalTok{, }\StringTok{"Market.rda"}\NormalTok{))}

\CommentTok{# Load the functions}
\KeywordTok{source}\NormalTok{(}\DataTypeTok{file =} \KeywordTok{here}\NormalTok{(}\StringTok{"Functions"}\NormalTok{, }\StringTok{"price_call.r"}\NormalTok{)) }\CommentTok{# Prices calls using the Black-Scholes formula}
\KeywordTok{source}\NormalTok{(}\DataTypeTok{file =} \KeywordTok{here}\NormalTok{(}\StringTok{"Functions"}\NormalTok{, }\StringTok{"lin_inter.r"}\NormalTok{))  }\CommentTok{# Linear interpolation of the interest rates}

\CommentTok{# Assign data to different variables}
\NormalTok{vix       <-}\StringTok{ }\KeywordTok{as.vector}\NormalTok{(Market}\OperatorTok{$}\NormalTok{vix)}
\NormalTok{sp_}\DecValTok{500}\NormalTok{    <-}\StringTok{ }\KeywordTok{as.vector}\NormalTok{(Market}\OperatorTok{$}\NormalTok{sp500)}
\NormalTok{calls     <-}\StringTok{ }\KeywordTok{as.vector}\NormalTok{(Market}\OperatorTok{$}\NormalTok{calls)}
\NormalTok{puts      <-}\StringTok{ }\KeywordTok{as.vector}\NormalTok{(Market}\OperatorTok{$}\NormalTok{puts)}

\CommentTok{# Create a matrix 'rates' with two columns: maturities and risk-free rates}
\NormalTok{rates     <-}\StringTok{ }\KeywordTok{matrix}\NormalTok{(}\DataTypeTok{data =} \OtherTok{NA}\NormalTok{, }\DataTypeTok{nrow =} \KeywordTok{length}\NormalTok{(Market}\OperatorTok{$}\NormalTok{rf), }\DataTypeTok{ncol =} \DecValTok{2}\NormalTok{)}
\NormalTok{rates[,}\DecValTok{1}\NormalTok{] <-}\StringTok{ }\KeywordTok{as.numeric}\NormalTok{((}\KeywordTok{attributes}\NormalTok{(Market}\OperatorTok{$}\NormalTok{rf))[[}\DecValTok{2}\NormalTok{]])}
\NormalTok{rates[,}\DecValTok{2}\NormalTok{] <-}\StringTok{ }\NormalTok{Market}\OperatorTok{$}\NormalTok{rf}

\CommentTok{# Assign column names to 'rates'}
\KeywordTok{colnames}\NormalTok{(rates) <-}\StringTok{ }\KeywordTok{c}\NormalTok{(}\StringTok{"Maturities"}\NormalTok{, }\StringTok{"Risk-free Rates"}\NormalTok{)}
\end{Highlighting}
\end{Shaded}

\hypertarget{part-iii-pricing-a-portfolio-of-options}{%
\section{Part III: Pricing a Portfolio of
Options}\label{part-iii-pricing-a-portfolio-of-options}}

The portfolio under consideration contains four options: K = 1600 and
T-t = 20 days, K = 1650 and T-t = 20 days, K = 1750 and T-t = 40 days,
and K = 1800 and T-t = 40 days. We assume that there is 250 days in a
year and thus convert times to expiry in years by dividing by 250. We
first create a matrix `book' that contains information regarding the
options in the portfolio.

\begin{Shaded}
\begin{Highlighting}[]
\CommentTok{# Create matrix}
\NormalTok{book_}\DecValTok{1}\NormalTok{ <-}\StringTok{ }\KeywordTok{matrix}\NormalTok{(}\DataTypeTok{data =} \OtherTok{NA}\NormalTok{, }\DataTypeTok{nrow =} \DecValTok{4}\NormalTok{, }\DataTypeTok{ncol =} \DecValTok{4}\NormalTok{)}

\CommentTok{# Assign names to columns}
\KeywordTok{colnames}\NormalTok{(book_}\DecValTok{1}\NormalTok{) <-}\StringTok{ }\KeywordTok{c}\NormalTok{(}\StringTok{"Quantity"}\NormalTok{, }\StringTok{"Call"}\NormalTok{, }\StringTok{"StrikePrice"}\NormalTok{, }\StringTok{"Maturity"}\NormalTok{)}

\CommentTok{# Store initial values}
\NormalTok{book_}\DecValTok{1}\NormalTok{[}\DecValTok{1}\NormalTok{,] <-}\StringTok{ }\KeywordTok{c}\NormalTok{(}\DecValTok{1}\NormalTok{, }\DecValTok{1}\NormalTok{, }\DecValTok{1600}\NormalTok{, }\DecValTok{20} \OperatorTok{/}\StringTok{ }\DecValTok{250}\NormalTok{)}
\NormalTok{book_}\DecValTok{1}\NormalTok{[}\DecValTok{2}\NormalTok{,] <-}\StringTok{ }\KeywordTok{c}\NormalTok{(}\DecValTok{1}\NormalTok{, }\DecValTok{1}\NormalTok{, }\DecValTok{1650}\NormalTok{, }\DecValTok{20} \OperatorTok{/}\StringTok{ }\DecValTok{250}\NormalTok{)}
\NormalTok{book_}\DecValTok{1}\NormalTok{[}\DecValTok{3}\NormalTok{,] <-}\StringTok{ }\KeywordTok{c}\NormalTok{(}\DecValTok{1}\NormalTok{, }\DecValTok{1}\NormalTok{, }\DecValTok{1750}\NormalTok{, }\DecValTok{40} \OperatorTok{/}\StringTok{ }\DecValTok{250}\NormalTok{)}
\NormalTok{book_}\DecValTok{1}\NormalTok{[}\DecValTok{4}\NormalTok{,] <-}\StringTok{ }\KeywordTok{c}\NormalTok{(}\DecValTok{1}\NormalTok{, }\DecValTok{1}\NormalTok{, }\DecValTok{1800}\NormalTok{, }\DecValTok{40} \OperatorTok{/}\StringTok{ }\DecValTok{250}\NormalTok{)}
\end{Highlighting}
\end{Shaded}

The next step is to use the most recent underlying asset price and VIX
value as well as the interpolated risk-free rates to compute the value
of each option in the portfolio that we store in the variable
`call\_price'. The value of the portfolio of options is simply the sum
of each option's value. To do this, we use two functions described
below.

\begin{enumerate}
\def\labelenumi{\arabic{enumi}.}
\tightlist
\item
  `lin\_inter': Takes as input a maturity in years (360-day basis) and
  outputs the associated rate. Uses linear interpolation with the given
  term structure.
\item
  `price\_call': Takes as inputs the underlying asset price, the option
  strike price, the risk-free rate, the volatility, and the maturity.
  Uses Black-Scholes model to price call options.
\end{enumerate}

\begin{Shaded}
\begin{Highlighting}[]
\CommentTok{# Number of observations}
\NormalTok{n_obs <-}\StringTok{ }\KeywordTok{length}\NormalTok{(sp_}\DecValTok{500}\NormalTok{)}

\CommentTok{# Convert 250-day basis year in 360-day basis year}
\NormalTok{m_}\DecValTok{1}\NormalTok{ <-}\StringTok{ }\KeywordTok{unique}\NormalTok{(book_}\DecValTok{1}\NormalTok{[,}\DecValTok{4}\NormalTok{])[}\DecValTok{1}\NormalTok{] }\OperatorTok{*}\StringTok{ }\NormalTok{(}\DecValTok{250} \OperatorTok{/}\StringTok{ }\DecValTok{360}\NormalTok{)}
\NormalTok{m_}\DecValTok{2}\NormalTok{ <-}\StringTok{ }\KeywordTok{unique}\NormalTok{(book_}\DecValTok{1}\NormalTok{[,}\DecValTok{4}\NormalTok{])[}\DecValTok{2}\NormalTok{] }\OperatorTok{*}\StringTok{ }\NormalTok{(}\DecValTok{250} \OperatorTok{/}\StringTok{ }\DecValTok{360}\NormalTok{)}

\CommentTok{# Interpolated interest rates}
\NormalTok{r_}\DecValTok{1}\NormalTok{ <-}\StringTok{ }\KeywordTok{lin_inter}\NormalTok{(m_}\DecValTok{1}\NormalTok{)}
\NormalTok{r_}\DecValTok{2}\NormalTok{ <-}\StringTok{ }\KeywordTok{lin_inter}\NormalTok{(m_}\DecValTok{2}\NormalTok{)}

\CommentTok{# Latest underlying asset price (spot price)}
\NormalTok{S_}\DecValTok{1}\NormalTok{ <-}\StringTok{ }\NormalTok{sp_}\DecValTok{500}\NormalTok{[n_obs]}

\CommentTok{# Latest VIX value}
\NormalTok{vol_}\DecValTok{1}\NormalTok{ <-}\StringTok{ }\NormalTok{vix[n_obs]}

\CommentTok{# Strike prices}
\NormalTok{K <-}\StringTok{ }\NormalTok{book_}\DecValTok{1}\NormalTok{[,}\DecValTok{3}\NormalTok{]}

\CommentTok{# Maturities}
\NormalTok{M <-}\StringTok{ }\NormalTok{book_}\DecValTok{1}\NormalTok{[,}\DecValTok{4}\NormalTok{]}

\CommentTok{# Initialize a vector to store call prices}
\NormalTok{call_price <-}\StringTok{ }\KeywordTok{c}\NormalTok{(}\KeywordTok{rep}\NormalTok{(}\OtherTok{NA}\NormalTok{, }\DecValTok{4}\NormalTok{))}

\CommentTok{# Compute the option values and store them in 'call_price'}
\NormalTok{call_price[}\DecValTok{1}\NormalTok{] <-}\StringTok{ }\KeywordTok{price_call}\NormalTok{(S_}\DecValTok{1}\NormalTok{, K[}\DecValTok{1}\NormalTok{], r_}\DecValTok{1}\NormalTok{, vol_}\DecValTok{1}\NormalTok{, M[}\DecValTok{1}\NormalTok{])}
\NormalTok{call_price[}\DecValTok{2}\NormalTok{] <-}\StringTok{ }\KeywordTok{price_call}\NormalTok{(S_}\DecValTok{1}\NormalTok{, K[}\DecValTok{2}\NormalTok{], r_}\DecValTok{1}\NormalTok{, vol_}\DecValTok{1}\NormalTok{, M[}\DecValTok{2}\NormalTok{])}
\NormalTok{call_price[}\DecValTok{3}\NormalTok{] <-}\StringTok{ }\KeywordTok{price_call}\NormalTok{(S_}\DecValTok{1}\NormalTok{, K[}\DecValTok{3}\NormalTok{], r_}\DecValTok{2}\NormalTok{, vol_}\DecValTok{1}\NormalTok{, M[}\DecValTok{3}\NormalTok{])}
\NormalTok{call_price[}\DecValTok{4}\NormalTok{] <-}\StringTok{ }\KeywordTok{price_call}\NormalTok{(S_}\DecValTok{1}\NormalTok{, K[}\DecValTok{4}\NormalTok{], r_}\DecValTok{2}\NormalTok{, vol_}\DecValTok{1}\NormalTok{, M[}\DecValTok{4}\NormalTok{])}

\CommentTok{# Compute the price of the portfolio and store it in 'PF_price_1'}
\NormalTok{PF_price_}\DecValTok{1}\NormalTok{ <-}\StringTok{ }\KeywordTok{sum}\NormalTok{(call_price)}
\end{Highlighting}
\end{Shaded}

\begin{verbatim}
## The value of the portfolio of European call options is 156.21$.
\end{verbatim}

\begin{verbatim}
## The options are worth respectively 87.31$, 47.52$, 15.10$, 6.28$.
\end{verbatim}

It is not surprising to see that the value of the first option is higher
than the value of the second option. Indeed, since the strike price is
lower for the first option, the first option is deeper ITM than the
second option and it should therefore have a higher price. The same
reasoning applies to the third and fourth options.

\hypertarget{part-iv-one-risk-driver-and-gaussian-model}{%
\section{Part IV: One Risk Driver and Gaussian
Model}\label{part-iv-one-risk-driver-and-gaussian-model}}

We now want to estimate the value-at-risk (VaR) and the expected
shortfall (ES) of this portfolio of options over the course of the
following week. To do so, we must first compute the underlying asset log
returns.

\begin{Shaded}
\begin{Highlighting}[]
\CommentTok{# Daily underlying asset log returns}
\NormalTok{log_return <-}\StringTok{ }\KeywordTok{diff}\NormalTok{(}\KeywordTok{log}\NormalTok{(sp_}\DecValTok{500}\NormalTok{))}
\end{Highlighting}
\end{Shaded}

Next, we assume that the underlying asset log returns follow a normal
distribution. The normal distribution parameters can be found by
computing the empirical mean and standard deviation of the underlying
asset daily log returns.

\begin{Shaded}
\begin{Highlighting}[]
\CommentTok{# Number of log returns}
\NormalTok{n_ret <-}\StringTok{ }\KeywordTok{length}\NormalTok{(log_return)}

\CommentTok{# Calibration}
\NormalTok{mu_hat <-}\StringTok{ }\KeywordTok{mean}\NormalTok{(log_return)}
\NormalTok{s2_hat <-}\StringTok{ }\KeywordTok{mean}\NormalTok{((log_return }\OperatorTok{-}\StringTok{ }\NormalTok{mu_hat)}\OperatorTok{^}\DecValTok{2}\NormalTok{) }\OperatorTok{*}\StringTok{ }\NormalTok{(n_ret }\OperatorTok{/}\StringTok{ }\NormalTok{(n_ret }\OperatorTok{-}\StringTok{ }\DecValTok{1}\NormalTok{))}
\NormalTok{sd_hat  <-}\StringTok{ }\NormalTok{s2_hat}\OperatorTok{^}\FloatTok{0.5}

\CommentTok{# Store parameters in 'theta_1'}
\NormalTok{theta_}\DecValTok{1}\NormalTok{ <-}\StringTok{ }\KeywordTok{c}\NormalTok{(mu_hat, sd_hat)}
\end{Highlighting}
\end{Shaded}

Now that we have estimated the parameters of the underlying asset return
log returns, we can run a simulation of the underlying asset price one
week ahead by generating normally distributed IID shocks with mean
`mu\_hat' and standard deviation `sd\_hat'. Since we found the
distribution parameters using daily log returns, we generate five shocks
per simulation and sum these shocks to obtain the overall shock over one
week. We fix the size of the simulation to 10 000.

\begin{Shaded}
\begin{Highlighting}[]
\CommentTok{# Number of simulation}
\NormalTok{H <-}\StringTok{ }\DecValTok{10000}

\CommentTok{# Number of days between now and the forecast horizon}
\NormalTok{t <-}\StringTok{ }\DecValTok{5}

\CommentTok{# Set seed for generating pseudo-random numbers}
\KeywordTok{set.seed}\NormalTok{(}\DecValTok{4321}\NormalTok{)}

\CommentTok{# Store in 'sim_ret_1' normally distributed IID shocks with mean 'mu_hat' and standard deviation 'sd_hat'}
\NormalTok{sim_ret_}\DecValTok{1}\NormalTok{ <-}\StringTok{ }\KeywordTok{matrix}\NormalTok{(}\KeywordTok{rnorm}\NormalTok{(t }\OperatorTok{*}\StringTok{ }\NormalTok{H, }\DataTypeTok{mean =}\NormalTok{ theta_}\DecValTok{1}\NormalTok{[}\DecValTok{1}\NormalTok{], }\DataTypeTok{sd =}\NormalTok{ theta_}\DecValTok{1}\NormalTok{[}\DecValTok{2}\NormalTok{]), }\DataTypeTok{nrow =}\NormalTok{ H, }\DataTypeTok{ncol =}\NormalTok{ t)}
\end{Highlighting}
\end{Shaded}

In order to obtain the underlying asset price one week from now, we
simply have to compute the exponential sum of the five daily normally
distributed IID shocks per trajectory and multiply by the latest
underlying asset price. The last thing we need to modify is the
risk-free rate for the remaining life of the option. The strike price
does not change over the option life and the volatility is assumed to be
constant.

\begin{Shaded}
\begin{Highlighting}[]
\CommentTok{# Compute the price of the underlying asset one week from now}
\NormalTok{S_}\DecValTok{2}\NormalTok{ <-}\StringTok{ }\NormalTok{S_}\DecValTok{1} \OperatorTok{*}\StringTok{ }\KeywordTok{exp}\NormalTok{(}\KeywordTok{rowSums}\NormalTok{(sim_ret_}\DecValTok{1}\NormalTok{))}

\CommentTok{# Update the risk-free rate (360-day basis year)}
\NormalTok{r_}\DecValTok{3}\NormalTok{   <-}\StringTok{ }\KeywordTok{lin_inter}\NormalTok{(m_}\DecValTok{1} \OperatorTok{-}\StringTok{ }\NormalTok{t }\OperatorTok{/}\StringTok{ }\DecValTok{360}\NormalTok{)}
\NormalTok{r_}\DecValTok{4}\NormalTok{   <-}\StringTok{ }\KeywordTok{lin_inter}\NormalTok{(m_}\DecValTok{2} \OperatorTok{-}\StringTok{ }\NormalTok{t }\OperatorTok{/}\StringTok{ }\DecValTok{360}\NormalTok{)}

\CommentTok{# Initialize a matrix to store call prices (10 000 rows (1 per simulation), 4 columns (1 per option))}
\NormalTok{call_price_}\DecValTok{2}\NormalTok{ <-}\StringTok{ }\KeywordTok{matrix}\NormalTok{(}\OtherTok{NA}\NormalTok{, }\DataTypeTok{nrow =}\NormalTok{ H, }\DataTypeTok{ncol =} \DecValTok{4}\NormalTok{)}

\CommentTok{# Loop through 10 000 simulations and price each call option}
\ControlFlowTok{for}\NormalTok{ (i }\ControlFlowTok{in} \DecValTok{1}\OperatorTok{:}\DecValTok{10000}\NormalTok{)\{}
\NormalTok{  call_price_}\DecValTok{2}\NormalTok{[i,}\DecValTok{1}\NormalTok{] <-}\StringTok{ }\KeywordTok{price_call}\NormalTok{(S_}\DecValTok{2}\NormalTok{[i], K[}\DecValTok{1}\NormalTok{], r_}\DecValTok{3}\NormalTok{, vol_}\DecValTok{1}\NormalTok{, M[}\DecValTok{1}\NormalTok{] }\OperatorTok{-}\StringTok{ }\NormalTok{t }\OperatorTok{/}\StringTok{ }\DecValTok{250}\NormalTok{)}
\NormalTok{  call_price_}\DecValTok{2}\NormalTok{[i,}\DecValTok{2}\NormalTok{] <-}\StringTok{ }\KeywordTok{price_call}\NormalTok{(S_}\DecValTok{2}\NormalTok{[i], K[}\DecValTok{2}\NormalTok{], r_}\DecValTok{3}\NormalTok{, vol_}\DecValTok{1}\NormalTok{, M[}\DecValTok{2}\NormalTok{] }\OperatorTok{-}\StringTok{ }\NormalTok{t }\OperatorTok{/}\StringTok{ }\DecValTok{250}\NormalTok{)}
\NormalTok{  call_price_}\DecValTok{2}\NormalTok{[i,}\DecValTok{3}\NormalTok{] <-}\StringTok{ }\KeywordTok{price_call}\NormalTok{(S_}\DecValTok{2}\NormalTok{[i], K[}\DecValTok{3}\NormalTok{], r_}\DecValTok{4}\NormalTok{, vol_}\DecValTok{1}\NormalTok{, M[}\DecValTok{3}\NormalTok{] }\OperatorTok{-}\StringTok{ }\NormalTok{t }\OperatorTok{/}\StringTok{ }\DecValTok{250}\NormalTok{)}
\NormalTok{  call_price_}\DecValTok{2}\NormalTok{[i,}\DecValTok{4}\NormalTok{] <-}\StringTok{ }\KeywordTok{price_call}\NormalTok{(S_}\DecValTok{2}\NormalTok{[i], K[}\DecValTok{4}\NormalTok{], r_}\DecValTok{4}\NormalTok{, vol_}\DecValTok{1}\NormalTok{, M[}\DecValTok{4}\NormalTok{] }\OperatorTok{-}\StringTok{ }\NormalTok{t }\OperatorTok{/}\StringTok{ }\DecValTok{250}\NormalTok{)}
\NormalTok{\}}
\end{Highlighting}
\end{Shaded}

For each replication, we can compute the value of the portfolio of
options by summing the value of the four call options. Recall that we
want to estimate the VaR and the ES of this portfolio of options over
the course of the following week. Therefore, for each replication, we
must compute a P\&L by discounting at the risk-free rate the value of
the portfolio of options one week from now and subtracting the value of
the portfolio observed today.

\begin{Shaded}
\begin{Highlighting}[]
\CommentTok{# Compute the price of the portfolio for each replication and store it in 'PF_price_2'}
\NormalTok{PF_price_}\DecValTok{2}\NormalTok{ <-}\StringTok{ }\KeywordTok{rowSums}\NormalTok{(call_price_}\DecValTok{2}\NormalTok{)}

\CommentTok{# Compute the risk-free rate for a period of 5 days (360-day basis year)}
\NormalTok{r_PL <-}\StringTok{ }\KeywordTok{lin_inter}\NormalTok{(t }\OperatorTok{/}\StringTok{ }\DecValTok{360}\NormalTok{)}

\CommentTok{# Compute the P&L}
\NormalTok{PL_}\DecValTok{1}\NormalTok{ <-}\StringTok{ }\NormalTok{PF_price_}\DecValTok{2} \OperatorTok{*}\StringTok{ }\KeywordTok{exp}\NormalTok{(}\OperatorTok{-}\NormalTok{(t }\OperatorTok{/}\StringTok{ }\DecValTok{360}\NormalTok{) }\OperatorTok{*}\StringTok{ }\NormalTok{r_PL) }\OperatorTok{-}\StringTok{ }\NormalTok{PF_price_}\DecValTok{1}
\end{Highlighting}
\end{Shaded}

The last step is to compute the VaR and the ES of the portfolio of
options P\&L distribution. In order to do so, we sort the P\&L values in
ascending order and identify the (1-alpha)-quartile of the distribution
for the VaR, where alpha is the risk level (in our case, 0.95). The ES
is simply the mean of the P\&L values smaller than the VaR.

\begin{Shaded}
\begin{Highlighting}[]
\CommentTok{# Set alpha to a desired significance level}
\NormalTok{alpha <-}\StringTok{ }\FloatTok{0.95}

\CommentTok{# Compute the VaR and the ES of the P&L distribution}
\NormalTok{VaR_}\DecValTok{1}\NormalTok{ <-}\StringTok{ }\KeywordTok{sort}\NormalTok{(PL_}\DecValTok{1}\NormalTok{)[(}\DecValTok{1} \OperatorTok{-}\StringTok{ }\NormalTok{alpha) }\OperatorTok{*}\StringTok{ }\NormalTok{H]}
\NormalTok{ES_}\DecValTok{1}\NormalTok{  <-}\StringTok{ }\KeywordTok{mean}\NormalTok{(}\KeywordTok{sort}\NormalTok{(PL_}\DecValTok{1}\NormalTok{)[}\DecValTok{1}\OperatorTok{:}\NormalTok{((}\DecValTok{1} \OperatorTok{-}\StringTok{ }\NormalTok{alpha) }\OperatorTok{*}\StringTok{ }\NormalTok{H)])}

\CommentTok{# Plot an histogram}
\KeywordTok{hist}\NormalTok{(PL_}\DecValTok{1}\NormalTok{, }\DataTypeTok{nclass =} \KeywordTok{round}\NormalTok{(}\DecValTok{10} \OperatorTok{*}\StringTok{ }\KeywordTok{log}\NormalTok{(}\KeywordTok{length}\NormalTok{(PL_}\DecValTok{1}\NormalTok{))), }\DataTypeTok{probability =} \OtherTok{TRUE}\NormalTok{)}

\CommentTok{# Add a vertical line to show the VaR}
\KeywordTok{abline}\NormalTok{(}\DataTypeTok{v   =} \KeywordTok{quantile}\NormalTok{(PL_}\DecValTok{1}\NormalTok{, }\DataTypeTok{probs =}\NormalTok{ (}\DecValTok{1} \OperatorTok{-}\StringTok{ }\NormalTok{alpha)),}
       \DataTypeTok{lty =} \DecValTok{1}\NormalTok{,}
       \DataTypeTok{lwd =} \FloatTok{2.5}\NormalTok{,}
       \DataTypeTok{col =} \StringTok{"red"}\NormalTok{)}
\end{Highlighting}
\end{Shaded}

\includegraphics{Project_Main_Notebook_files/figure-latex/unnamed-chunk-10-1.pdf}

\begin{verbatim}
## The value at risk at alpha = 0.95 is -121.67$.
\end{verbatim}

\begin{verbatim}
## The expected shortfall at alpha = 0.95 is -134.17$.
\end{verbatim}

{[}Discussion on the result (talk about asymmetry of call option
payoffs) and the assumption (normal distribution, volatility is
constant){]}

\hypertarget{part-v-two-risk-drivers-and-gaussian-model}{%
\section{Part V: Two risk drivers and Gaussian
model}\label{part-v-two-risk-drivers-and-gaussian-model}}

\begin{Shaded}
\begin{Highlighting}[]
\CommentTok{# install.packages("MASS")}
\KeywordTok{library}\NormalTok{(}\StringTok{"MASS"}\NormalTok{)}

\CommentTok{# Daily VIX log returns}
\NormalTok{vix_return <-}\StringTok{ }\KeywordTok{diff}\NormalTok{(}\KeywordTok{log}\NormalTok{(vix))}

\CommentTok{# Store underlying asset log returns and VIX log returns in 'rets'}
\NormalTok{rets     <-}\StringTok{ }\KeywordTok{matrix}\NormalTok{(}\OtherTok{NA}\NormalTok{, }\DataTypeTok{nrow =}\NormalTok{ n_ret, }\DataTypeTok{ncol =} \DecValTok{2}\NormalTok{)}
\NormalTok{rets[,}\DecValTok{1}\NormalTok{] <-}\StringTok{ }\NormalTok{log_return}
\NormalTok{rets[,}\DecValTok{2}\NormalTok{] <-}\StringTok{ }\NormalTok{vix_return}

\CommentTok{# Calibration}
\NormalTok{mu_hat <-}\StringTok{ }\KeywordTok{colMeans}\NormalTok{(rets)}
\NormalTok{sg_hat <-}\StringTok{ }\NormalTok{((n_ret }\OperatorTok{-}\StringTok{ }\DecValTok{1}\NormalTok{) }\OperatorTok{/}\StringTok{ }\NormalTok{n_ret) }\OperatorTok{*}\StringTok{ }\KeywordTok{cov}\NormalTok{(rets)}

\CommentTok{# Store parameters in 'theta_2'}
\NormalTok{theta_}\DecValTok{2}\NormalTok{ <-}\StringTok{ }\KeywordTok{list}\NormalTok{(}\DataTypeTok{mu =}\NormalTok{ mu_hat, }\DataTypeTok{sigma =}\NormalTok{ sg_hat)}

\CommentTok{# Initialize the array 'sim_ret_2'}
\NormalTok{sim_ret_}\DecValTok{2}\NormalTok{ <-}\StringTok{ }\KeywordTok{array}\NormalTok{(}\DataTypeTok{data =} \OtherTok{NA}\NormalTok{, }\DataTypeTok{dim =} \KeywordTok{c}\NormalTok{(H, }\DecValTok{2}\NormalTok{, t))}

\CommentTok{# Set seed for generating pseudo-random numbers}
\KeywordTok{set.seed}\NormalTok{(}\DecValTok{4321}\NormalTok{)}

\CommentTok{# Store in 'sim_ret_2' daily shocks from a multivariate normal distribution with parameters 'theta_2'}
\ControlFlowTok{for}\NormalTok{ (i }\ControlFlowTok{in} \DecValTok{1}\OperatorTok{:}\NormalTok{t) \{}
\NormalTok{  sim_ret_}\DecValTok{2}\NormalTok{[,,i] <-}\StringTok{ }\KeywordTok{mvrnorm}\NormalTok{(}\DataTypeTok{n =}\NormalTok{ H, }\DataTypeTok{mu =}\NormalTok{ theta_}\DecValTok{2}\OperatorTok{$}\NormalTok{mu, }\DataTypeTok{Sigma =}\NormalTok{ theta_}\DecValTok{2}\OperatorTok{$}\NormalTok{sigma)}
\NormalTok{\}}

\CommentTok{# Initialize two vectors that contain the underlying asset price and the VIX value one week from now}
\NormalTok{S_}\DecValTok{3}\NormalTok{   <-}\StringTok{ }\KeywordTok{rep}\NormalTok{(}\OtherTok{NA}\NormalTok{, H)}
\NormalTok{vol_}\DecValTok{3}\NormalTok{ <-}\StringTok{ }\KeywordTok{rep}\NormalTok{(}\OtherTok{NA}\NormalTok{, H)}

\CommentTok{# Compute the underlying asset price and the VIX value one week from now}
\ControlFlowTok{for}\NormalTok{ (i }\ControlFlowTok{in} \DecValTok{1}\OperatorTok{:}\NormalTok{H) \{}
\NormalTok{  S_}\DecValTok{3}\NormalTok{[i]   <-}\StringTok{ }\NormalTok{S_}\DecValTok{1} \OperatorTok{*}\StringTok{ }\KeywordTok{exp}\NormalTok{(}\KeywordTok{sum}\NormalTok{(sim_ret_}\DecValTok{2}\NormalTok{[i,}\DecValTok{1}\NormalTok{,]))}
\NormalTok{  vol_}\DecValTok{3}\NormalTok{[i] <-}\StringTok{ }\NormalTok{vol_}\DecValTok{1} \OperatorTok{*}\StringTok{ }\KeywordTok{exp}\NormalTok{(}\KeywordTok{sum}\NormalTok{(sim_ret_}\DecValTok{2}\NormalTok{[i,}\DecValTok{2}\NormalTok{,]))}
\NormalTok{\}}

\CommentTok{# Initialize a matrix to store call prices (10 000 rows (1 per simulation), 4 columns (1 per option))}
\NormalTok{call_price_}\DecValTok{3}\NormalTok{ <-}\StringTok{ }\KeywordTok{matrix}\NormalTok{(}\OtherTok{NA}\NormalTok{, }\DataTypeTok{nrow =}\NormalTok{ H, }\DataTypeTok{ncol =} \DecValTok{4}\NormalTok{)}

\CommentTok{# Loop through 10 000 simulations and price each call option}
\ControlFlowTok{for}\NormalTok{ (i }\ControlFlowTok{in} \DecValTok{1}\OperatorTok{:}\DecValTok{10000}\NormalTok{)\{}
\NormalTok{  call_price_}\DecValTok{3}\NormalTok{[i,}\DecValTok{1}\NormalTok{] <-}\StringTok{ }\KeywordTok{price_call}\NormalTok{(S_}\DecValTok{3}\NormalTok{[i], K[}\DecValTok{1}\NormalTok{], r_}\DecValTok{3}\NormalTok{, vol_}\DecValTok{3}\NormalTok{[i], M[}\DecValTok{1}\NormalTok{] }\OperatorTok{-}\StringTok{ }\NormalTok{t }\OperatorTok{/}\StringTok{ }\DecValTok{250}\NormalTok{)}
\NormalTok{  call_price_}\DecValTok{3}\NormalTok{[i,}\DecValTok{2}\NormalTok{] <-}\StringTok{ }\KeywordTok{price_call}\NormalTok{(S_}\DecValTok{3}\NormalTok{[i], K[}\DecValTok{2}\NormalTok{], r_}\DecValTok{3}\NormalTok{, vol_}\DecValTok{3}\NormalTok{[i], M[}\DecValTok{2}\NormalTok{] }\OperatorTok{-}\StringTok{ }\NormalTok{t }\OperatorTok{/}\StringTok{ }\DecValTok{250}\NormalTok{)}
\NormalTok{  call_price_}\DecValTok{3}\NormalTok{[i,}\DecValTok{3}\NormalTok{] <-}\StringTok{ }\KeywordTok{price_call}\NormalTok{(S_}\DecValTok{3}\NormalTok{[i], K[}\DecValTok{3}\NormalTok{], r_}\DecValTok{4}\NormalTok{, vol_}\DecValTok{3}\NormalTok{[i], M[}\DecValTok{3}\NormalTok{] }\OperatorTok{-}\StringTok{ }\NormalTok{t }\OperatorTok{/}\StringTok{ }\DecValTok{250}\NormalTok{)}
\NormalTok{  call_price_}\DecValTok{3}\NormalTok{[i,}\DecValTok{4}\NormalTok{] <-}\StringTok{ }\KeywordTok{price_call}\NormalTok{(S_}\DecValTok{3}\NormalTok{[i], K[}\DecValTok{4}\NormalTok{], r_}\DecValTok{4}\NormalTok{, vol_}\DecValTok{3}\NormalTok{[i], M[}\DecValTok{4}\NormalTok{] }\OperatorTok{-}\StringTok{ }\NormalTok{t }\OperatorTok{/}\StringTok{ }\DecValTok{250}\NormalTok{)}
\NormalTok{\}}

\CommentTok{# Compute the price of the portfolio for each replication and store it in 'PF_price_3'}
\NormalTok{PF_price_}\DecValTok{3}\NormalTok{ <-}\StringTok{ }\KeywordTok{rowSums}\NormalTok{(call_price_}\DecValTok{3}\NormalTok{)}

\CommentTok{# Compute the P&L}
\NormalTok{PL_}\DecValTok{2}\NormalTok{ <-}\StringTok{ }\NormalTok{PF_price_}\DecValTok{3} \OperatorTok{*}\StringTok{ }\KeywordTok{exp}\NormalTok{(}\OperatorTok{-}\NormalTok{(t }\OperatorTok{/}\StringTok{ }\DecValTok{360}\NormalTok{) }\OperatorTok{*}\StringTok{ }\NormalTok{r_PL) }\OperatorTok{-}\StringTok{ }\NormalTok{PF_price_}\DecValTok{1}

\CommentTok{# Compute the VaR and the ES of the P&L distribution}
\NormalTok{VaR_}\DecValTok{2}\NormalTok{ <-}\StringTok{ }\KeywordTok{sort}\NormalTok{(PL_}\DecValTok{2}\NormalTok{)[(}\DecValTok{1} \OperatorTok{-}\StringTok{ }\NormalTok{alpha) }\OperatorTok{*}\StringTok{ }\NormalTok{H]}
\NormalTok{ES_}\DecValTok{2}\NormalTok{  <-}\StringTok{ }\KeywordTok{mean}\NormalTok{(}\KeywordTok{sort}\NormalTok{(PL_}\DecValTok{2}\NormalTok{)[}\DecValTok{1}\OperatorTok{:}\NormalTok{((}\DecValTok{1} \OperatorTok{-}\StringTok{ }\NormalTok{alpha) }\OperatorTok{*}\StringTok{ }\NormalTok{H)])}

\CommentTok{# Plot an histogram}
\KeywordTok{hist}\NormalTok{(PL_}\DecValTok{2}\NormalTok{, }\DataTypeTok{nclass =} \KeywordTok{round}\NormalTok{(}\DecValTok{10} \OperatorTok{*}\StringTok{ }\KeywordTok{log}\NormalTok{(}\KeywordTok{length}\NormalTok{(PL_}\DecValTok{2}\NormalTok{))), }\DataTypeTok{probability =} \OtherTok{TRUE}\NormalTok{)}

\CommentTok{# Add a vertical line to show the VaR}
\KeywordTok{abline}\NormalTok{(}\DataTypeTok{v   =} \KeywordTok{quantile}\NormalTok{(PL_}\DecValTok{2}\NormalTok{, }\DataTypeTok{probs =}\NormalTok{ (}\DecValTok{1} \OperatorTok{-}\StringTok{ }\NormalTok{alpha)),}
       \DataTypeTok{lty =} \DecValTok{1}\NormalTok{,}
       \DataTypeTok{lwd =} \FloatTok{2.5}\NormalTok{,}
       \DataTypeTok{col =} \StringTok{"red"}\NormalTok{)}
\end{Highlighting}
\end{Shaded}

\includegraphics{Project_Main_Notebook_files/figure-latex/unnamed-chunk-12-1.pdf}

\begin{verbatim}
## The value at risk at alpha = 0.95 is -111.15$.
\end{verbatim}

\begin{verbatim}
## The expected shortfall at alpha = 0.95 is -123.77$.
\end{verbatim}

{[}Discussion on the result (VaR is smaller because the underlying asset
and the VIX are negatively correlated; when the value of the underlying
asset drops, the volatility tends to rise and offsets the effect of the
underlying asset price on the call option price) and the assumption
(normal distribution for log returns){]}

\hypertarget{part-vi-two-risk-drivers-and-copula-marginal-model}{%
\section{Part VI: Two risk drivers and copula-marginal
model}\label{part-vi-two-risk-drivers-and-copula-marginal-model}}

\begin{Shaded}
\begin{Highlighting}[]
\CommentTok{# install.packages("copula")}
\KeywordTok{library}\NormalTok{(}\StringTok{"copula"}\NormalTok{)}

\CommentTok{# install.packages("fGarch")}
\KeywordTok{library}\NormalTok{(}\StringTok{"fGarch"}\NormalTok{)}
\end{Highlighting}
\end{Shaded}

\begin{verbatim}
## Loading required package: timeDate
\end{verbatim}

\begin{verbatim}
## Loading required package: timeSeries
\end{verbatim}

\begin{verbatim}
## Loading required package: fBasics
\end{verbatim}

\begin{Shaded}
\begin{Highlighting}[]
\KeywordTok{library}\NormalTok{(}\StringTok{"timeDate"}\NormalTok{)}
\KeywordTok{library}\NormalTok{(}\StringTok{"timeSeries"}\NormalTok{)}
\KeywordTok{library}\NormalTok{(}\StringTok{"fBasics"}\NormalTok{)}

\CommentTok{# Load the function}
\KeywordTok{source}\NormalTok{(}\DataTypeTok{file =} \KeywordTok{here}\NormalTok{(}\StringTok{"Functions"}\NormalTok{, }\StringTok{"nll_student.r"}\NormalTok{)) }\CommentTok{# Computes the negative log-likelihood function}

\CommentTok{# Define an initial vector of parameters for the underlying asset log returns}
\NormalTok{theta_}\DecValTok{0}\NormalTok{ <-}\StringTok{ }\KeywordTok{c}\NormalTok{(}\KeywordTok{mean}\NormalTok{(log_return), }\KeywordTok{sd}\NormalTok{(log_return), }\DecValTok{10}\NormalTok{)}

\CommentTok{# Calibrate the student-t distribution on the underlying asset log returns}
\NormalTok{tmp <-}\StringTok{ }\KeywordTok{optim}\NormalTok{(}\DataTypeTok{par =}\NormalTok{ theta_}\DecValTok{0}\NormalTok{,}
             \DataTypeTok{fn =}\NormalTok{ nll_student,}
             \DataTypeTok{method =} \StringTok{"L-BFGS-B"}\NormalTok{,}
             \DataTypeTok{lower =} \KeywordTok{c}\NormalTok{(}\OperatorTok{-}\OtherTok{Inf}\NormalTok{, }\FloatTok{1e-5}\NormalTok{, }\DecValTok{10}\NormalTok{),}
             \DataTypeTok{x =}\NormalTok{ log_return)}

\CommentTok{# Store parameters in 'theta_3'}
\NormalTok{theta_}\DecValTok{3}\NormalTok{ <-}\StringTok{ }\NormalTok{tmp}\OperatorTok{$}\NormalTok{par}

\CommentTok{# Define an initial vector of parameters for the VIX log returns}
\NormalTok{theta_}\DecValTok{0}\NormalTok{ <-}\StringTok{ }\KeywordTok{c}\NormalTok{(}\KeywordTok{mean}\NormalTok{(vix_return), }\KeywordTok{sd}\NormalTok{(vix_return), }\DecValTok{5}\NormalTok{)}

\CommentTok{# Calibrate the student-t distribution on the VIX log returns}
\NormalTok{tmp <-}\StringTok{ }\KeywordTok{optim}\NormalTok{(}\DataTypeTok{par =}\NormalTok{ theta_}\DecValTok{0}\NormalTok{,}
             \DataTypeTok{fn =}\NormalTok{ nll_student,}
             \DataTypeTok{method =} \StringTok{"L-BFGS-B"}\NormalTok{,}
             \DataTypeTok{lower =} \KeywordTok{c}\NormalTok{(}\OperatorTok{-}\OtherTok{Inf}\NormalTok{, }\FloatTok{1e-5}\NormalTok{, }\DecValTok{5}\NormalTok{),}
             \DataTypeTok{x =}\NormalTok{ vix_return)}

\CommentTok{# Store parameters in 'theta_3'}
\NormalTok{theta_}\DecValTok{4}\NormalTok{ <-}\StringTok{ }\NormalTok{tmp}\OperatorTok{$}\NormalTok{par}

\CommentTok{# Compute 'U_1' and 'U_2' and combine these two variables in 'U'}
\NormalTok{U_}\DecValTok{1}\NormalTok{ <-}\StringTok{ }\KeywordTok{pstd}\NormalTok{(log_return, }\DataTypeTok{mean =}\NormalTok{ theta_}\DecValTok{3}\NormalTok{[}\DecValTok{1}\NormalTok{], }\DataTypeTok{sd =}\NormalTok{ theta_}\DecValTok{3}\NormalTok{[}\DecValTok{2}\NormalTok{], }\DataTypeTok{nu =}\NormalTok{ theta_}\DecValTok{3}\NormalTok{[}\DecValTok{3}\NormalTok{])}
\NormalTok{U_}\DecValTok{2}\NormalTok{ <-}\StringTok{ }\KeywordTok{pstd}\NormalTok{(log_return, }\DataTypeTok{mean =}\NormalTok{ theta_}\DecValTok{4}\NormalTok{[}\DecValTok{1}\NormalTok{], }\DataTypeTok{sd =}\NormalTok{ theta_}\DecValTok{4}\NormalTok{[}\DecValTok{2}\NormalTok{], }\DataTypeTok{nu =}\NormalTok{ theta_}\DecValTok{4}\NormalTok{[}\DecValTok{3}\NormalTok{])}
\NormalTok{U   <-}\StringTok{ }\KeywordTok{cbind}\NormalTok{(U_}\DecValTok{1}\NormalTok{, U_}\DecValTok{2}\NormalTok{)}

\CommentTok{# Calibrate a Gaussian copula}
\NormalTok{C   <-}\StringTok{ }\KeywordTok{normalCopula}\NormalTok{(}\DataTypeTok{dim =} \DecValTok{2}\NormalTok{)}
\NormalTok{fit <-}\StringTok{ }\KeywordTok{fitCopula}\NormalTok{(C, }\DataTypeTok{data =}\NormalTok{ U, }\DataTypeTok{method =} \StringTok{"ml"}\NormalTok{)}

\CommentTok{# Set seed for generating pseudo-random numbers}
\KeywordTok{set.seed}\NormalTok{(}\DecValTok{4321}\NormalTok{)}

\NormalTok{sim_U          <-}\StringTok{ }\KeywordTok{rCopula}\NormalTok{(H }\OperatorTok{*}\StringTok{ }\NormalTok{t, fit}\OperatorTok{@}\NormalTok{copula)}
\NormalTok{sim_log_return <-}\StringTok{ }\KeywordTok{qstd}\NormalTok{(sim_U[,}\DecValTok{1}\NormalTok{], }\DataTypeTok{mean =}\NormalTok{ theta_}\DecValTok{3}\NormalTok{[}\DecValTok{1}\NormalTok{], }\DataTypeTok{sd =}\NormalTok{ theta_}\DecValTok{3}\NormalTok{[}\DecValTok{2}\NormalTok{], }\DataTypeTok{nu =}\NormalTok{ theta_}\DecValTok{3}\NormalTok{[}\DecValTok{3}\NormalTok{])}
\NormalTok{sim_vix_return <-}\StringTok{ }\KeywordTok{qstd}\NormalTok{(sim_U[,}\DecValTok{2}\NormalTok{], }\DataTypeTok{mean =}\NormalTok{ theta_}\DecValTok{4}\NormalTok{[}\DecValTok{1}\NormalTok{], }\DataTypeTok{sd =}\NormalTok{ theta_}\DecValTok{4}\NormalTok{[}\DecValTok{2}\NormalTok{], }\DataTypeTok{nu =}\NormalTok{ theta_}\DecValTok{4}\NormalTok{[}\DecValTok{3}\NormalTok{])}

\CommentTok{# Initialize the array 'sim_ret_3'}
\NormalTok{sim_ret_}\DecValTok{3}\NormalTok{ <-}\StringTok{ }\KeywordTok{array}\NormalTok{(}\DataTypeTok{data =} \OtherTok{NA}\NormalTok{, }\DataTypeTok{dim =} \KeywordTok{c}\NormalTok{(H, }\DecValTok{2}\NormalTok{, t))}

\CommentTok{# Store in 'sim_ret_3' daily log returns for the underlying asset and the VIX}
\ControlFlowTok{for}\NormalTok{ (i }\ControlFlowTok{in} \DecValTok{1}\OperatorTok{:}\NormalTok{t) \{}
\NormalTok{  sim_ret_}\DecValTok{3}\NormalTok{[,,i] <-}\StringTok{ }\KeywordTok{c}\NormalTok{(sim_log_return[(H }\OperatorTok{*}\StringTok{ }\NormalTok{(i }\OperatorTok{-}\StringTok{ }\DecValTok{1}\NormalTok{) }\OperatorTok{+}\StringTok{ }\DecValTok{1}\NormalTok{)}\OperatorTok{:}\NormalTok{(H }\OperatorTok{*}\StringTok{ }\NormalTok{i)], sim_vix_return[(H }\OperatorTok{*}\StringTok{ }\NormalTok{(i }\OperatorTok{-}\StringTok{ }\DecValTok{1}\NormalTok{) }\OperatorTok{+}\StringTok{ }\DecValTok{1}\NormalTok{)}\OperatorTok{:}\NormalTok{(H }\OperatorTok{*}\StringTok{ }\NormalTok{i)])}
\NormalTok{\}}

\CommentTok{# Initialize two vectors that contain the underlying asset price and the VIX value one week from now}
\NormalTok{S_}\DecValTok{4}\NormalTok{   <-}\StringTok{ }\KeywordTok{rep}\NormalTok{(}\OtherTok{NA}\NormalTok{, H)}
\NormalTok{vol_}\DecValTok{4}\NormalTok{ <-}\StringTok{ }\KeywordTok{rep}\NormalTok{(}\OtherTok{NA}\NormalTok{, H)}

\CommentTok{# Compute the underlying asset price and the VIX value one week from now}
\ControlFlowTok{for}\NormalTok{ (i }\ControlFlowTok{in} \DecValTok{1}\OperatorTok{:}\NormalTok{H) \{}
\NormalTok{  S_}\DecValTok{4}\NormalTok{[i]   <-}\StringTok{ }\NormalTok{S_}\DecValTok{1} \OperatorTok{*}\StringTok{ }\KeywordTok{exp}\NormalTok{(}\KeywordTok{sum}\NormalTok{(sim_ret_}\DecValTok{3}\NormalTok{[i,}\DecValTok{1}\NormalTok{,]))}
\NormalTok{  vol_}\DecValTok{4}\NormalTok{[i] <-}\StringTok{ }\NormalTok{vol_}\DecValTok{1} \OperatorTok{*}\StringTok{ }\KeywordTok{exp}\NormalTok{(}\KeywordTok{sum}\NormalTok{(sim_ret_}\DecValTok{3}\NormalTok{[i,}\DecValTok{2}\NormalTok{,]))}
\NormalTok{\}}

\CommentTok{# Initialize a matrix to store call prices (10 000 rows (1 per simulation), 4 columns (1 per option))}
\NormalTok{call_price_}\DecValTok{4}\NormalTok{ <-}\StringTok{ }\KeywordTok{matrix}\NormalTok{(}\OtherTok{NA}\NormalTok{, }\DataTypeTok{nrow =}\NormalTok{ H, }\DataTypeTok{ncol =} \DecValTok{4}\NormalTok{)}

\CommentTok{# Loop through 10 000 simulations and price each call option}
\ControlFlowTok{for}\NormalTok{ (i }\ControlFlowTok{in} \DecValTok{1}\OperatorTok{:}\DecValTok{10000}\NormalTok{)\{}
\NormalTok{  call_price_}\DecValTok{4}\NormalTok{[i,}\DecValTok{1}\NormalTok{] <-}\StringTok{ }\KeywordTok{price_call}\NormalTok{(S_}\DecValTok{4}\NormalTok{[i], K[}\DecValTok{1}\NormalTok{], r_}\DecValTok{3}\NormalTok{, vol_}\DecValTok{4}\NormalTok{[i], M[}\DecValTok{1}\NormalTok{] }\OperatorTok{-}\StringTok{ }\NormalTok{t }\OperatorTok{/}\StringTok{ }\DecValTok{250}\NormalTok{)}
\NormalTok{  call_price_}\DecValTok{4}\NormalTok{[i,}\DecValTok{2}\NormalTok{] <-}\StringTok{ }\KeywordTok{price_call}\NormalTok{(S_}\DecValTok{4}\NormalTok{[i], K[}\DecValTok{2}\NormalTok{], r_}\DecValTok{3}\NormalTok{, vol_}\DecValTok{4}\NormalTok{[i], M[}\DecValTok{2}\NormalTok{] }\OperatorTok{-}\StringTok{ }\NormalTok{t }\OperatorTok{/}\StringTok{ }\DecValTok{250}\NormalTok{)}
\NormalTok{  call_price_}\DecValTok{4}\NormalTok{[i,}\DecValTok{3}\NormalTok{] <-}\StringTok{ }\KeywordTok{price_call}\NormalTok{(S_}\DecValTok{4}\NormalTok{[i], K[}\DecValTok{3}\NormalTok{], r_}\DecValTok{4}\NormalTok{, vol_}\DecValTok{4}\NormalTok{[i], M[}\DecValTok{3}\NormalTok{] }\OperatorTok{-}\StringTok{ }\NormalTok{t }\OperatorTok{/}\StringTok{ }\DecValTok{250}\NormalTok{)}
\NormalTok{  call_price_}\DecValTok{4}\NormalTok{[i,}\DecValTok{4}\NormalTok{] <-}\StringTok{ }\KeywordTok{price_call}\NormalTok{(S_}\DecValTok{4}\NormalTok{[i], K[}\DecValTok{4}\NormalTok{], r_}\DecValTok{4}\NormalTok{, vol_}\DecValTok{4}\NormalTok{[i], M[}\DecValTok{4}\NormalTok{] }\OperatorTok{-}\StringTok{ }\NormalTok{t }\OperatorTok{/}\StringTok{ }\DecValTok{250}\NormalTok{)}
\NormalTok{\}}

\CommentTok{# Compute the price of the portfolio for each replication and store it in 'PF_price_4'}
\NormalTok{PF_price_}\DecValTok{4}\NormalTok{ <-}\StringTok{ }\KeywordTok{rowSums}\NormalTok{(call_price_}\DecValTok{4}\NormalTok{)}

\CommentTok{# Compute the P&L}
\NormalTok{PL_}\DecValTok{3}\NormalTok{ <-}\StringTok{ }\NormalTok{PF_price_}\DecValTok{4} \OperatorTok{*}\StringTok{ }\KeywordTok{exp}\NormalTok{(}\OperatorTok{-}\NormalTok{(t }\OperatorTok{/}\StringTok{ }\DecValTok{360}\NormalTok{) }\OperatorTok{*}\StringTok{ }\NormalTok{r_PL) }\OperatorTok{-}\StringTok{ }\NormalTok{PF_price_}\DecValTok{1}

\CommentTok{# Compute the VaR and the ES of the P&L distribution}
\NormalTok{VaR_}\DecValTok{3}\NormalTok{ <-}\StringTok{ }\KeywordTok{sort}\NormalTok{(PL_}\DecValTok{3}\NormalTok{)[(}\DecValTok{1} \OperatorTok{-}\StringTok{ }\NormalTok{alpha) }\OperatorTok{*}\StringTok{ }\NormalTok{H]}
\NormalTok{ES_}\DecValTok{3}\NormalTok{  <-}\StringTok{ }\KeywordTok{mean}\NormalTok{(}\KeywordTok{sort}\NormalTok{(PL_}\DecValTok{3}\NormalTok{)[}\DecValTok{1}\OperatorTok{:}\NormalTok{((}\DecValTok{1} \OperatorTok{-}\StringTok{ }\NormalTok{alpha) }\OperatorTok{*}\StringTok{ }\NormalTok{H)])}

\CommentTok{# Plot an histogram}
\KeywordTok{hist}\NormalTok{(PL_}\DecValTok{3}\NormalTok{, }\DataTypeTok{nclass =} \KeywordTok{round}\NormalTok{(}\DecValTok{10} \OperatorTok{*}\StringTok{ }\KeywordTok{log}\NormalTok{(}\KeywordTok{length}\NormalTok{(PL_}\DecValTok{3}\NormalTok{))), }\DataTypeTok{probability =} \OtherTok{TRUE}\NormalTok{)}

\CommentTok{# Add a vertical line to show the VaR}
\KeywordTok{abline}\NormalTok{(}\DataTypeTok{v   =} \KeywordTok{quantile}\NormalTok{(PL_}\DecValTok{3}\NormalTok{, }\DataTypeTok{probs =}\NormalTok{ (}\DecValTok{1} \OperatorTok{-}\StringTok{ }\NormalTok{alpha)),}
       \DataTypeTok{lty =} \DecValTok{1}\NormalTok{,}
       \DataTypeTok{lwd =} \FloatTok{2.5}\NormalTok{,}
       \DataTypeTok{col =} \StringTok{"red"}\NormalTok{)}
\end{Highlighting}
\end{Shaded}

\includegraphics{Project_Main_Notebook_files/figure-latex/unnamed-chunk-14-1.pdf}

\begin{verbatim}
## The value at risk at alpha = 0.95 is -112.92$.
\end{verbatim}

\begin{verbatim}
## The expected shortfall at alpha = 0.95 is -127.30$.
\end{verbatim}

\hypertarget{part-vii-volatility-surface}{%
\section{Part VII: Volatility
Surface}\label{part-vii-volatility-surface}}

\begin{Shaded}
\begin{Highlighting}[]
\CommentTok{#install.packages("rgl")}
\KeywordTok{library}\NormalTok{(}\StringTok{"rgl"}\NormalTok{)}

\CommentTok{# Load the functions}
\KeywordTok{source}\NormalTok{(}\DataTypeTok{file =} \KeywordTok{here}\NormalTok{(}\StringTok{"Functions"}\NormalTok{, }\StringTok{"vol_surface.R"}\NormalTok{))   }\CommentTok{# Implied volatility of an option}
\KeywordTok{source}\NormalTok{(}\DataTypeTok{file =} \KeywordTok{here}\NormalTok{(}\StringTok{"Functions"}\NormalTok{, }\StringTok{"vol_calibrate.r"}\NormalTok{)) }\CommentTok{# Sum of absolute deviations of implied volalities}

\CommentTok{# Count the number of traded options}
\NormalTok{nb_opts <-}\StringTok{ }\KeywordTok{nrow}\NormalTok{(Market}\OperatorTok{$}\NormalTok{calls) }\OperatorTok{+}\StringTok{ }\KeywordTok{nrow}\NormalTok{(Market}\OperatorTok{$}\NormalTok{puts)}

\CommentTok{# Build a matrix that contains the information relevant to traded call and put options}
\NormalTok{mkt_vol <-}\StringTok{ }\KeywordTok{matrix}\NormalTok{(}\OtherTok{NA}\NormalTok{, }\DataTypeTok{nrow =}\NormalTok{ nb_opts, }\DataTypeTok{ncol =} \DecValTok{4}\NormalTok{)}

\CommentTok{# Assign names to columns}
\KeywordTok{colnames}\NormalTok{(mkt_vol) <-}\StringTok{ }\KeywordTok{c}\NormalTok{(}\StringTok{"S"}\NormalTok{, }\StringTok{"K"}\NormalTok{, }\StringTok{"tau"}\NormalTok{, }\StringTok{"IV"}\NormalTok{)}

\CommentTok{# Latest underlying asset price (spot price)}
\NormalTok{mkt_vol[,}\DecValTok{1}\NormalTok{] <-}\StringTok{ }\KeywordTok{t}\NormalTok{(}\KeywordTok{t}\NormalTok{(}\KeywordTok{rep}\NormalTok{(sp_}\DecValTok{500}\NormalTok{[n_obs], nb_opts)))}

\CommentTok{# Strike price of call options}
\NormalTok{mkt_vol[}\DecValTok{1}\OperatorTok{:}\KeywordTok{nrow}\NormalTok{(Market}\OperatorTok{$}\NormalTok{calls),}\DecValTok{2}\NormalTok{] <-}\StringTok{ }\NormalTok{Market}\OperatorTok{$}\NormalTok{calls[,}\DecValTok{1}\NormalTok{]}

\CommentTok{# Strike price of put options}
\NormalTok{mkt_vol[(}\KeywordTok{nrow}\NormalTok{(Market}\OperatorTok{$}\NormalTok{calls)}\OperatorTok{+}\DecValTok{1}\NormalTok{)}\OperatorTok{:}\NormalTok{nb_opts,}\DecValTok{2}\NormalTok{] <-}\StringTok{ }\NormalTok{Market}\OperatorTok{$}\NormalTok{puts[,}\DecValTok{1}\NormalTok{]}

\CommentTok{# Time to expiry of call options}
\NormalTok{mkt_vol[}\DecValTok{1}\OperatorTok{:}\KeywordTok{nrow}\NormalTok{(Market}\OperatorTok{$}\NormalTok{calls),}\DecValTok{3}\NormalTok{] <-}\StringTok{ }\NormalTok{Market}\OperatorTok{$}\NormalTok{calls[,}\DecValTok{2}\NormalTok{]}

\CommentTok{# Time to expiry of put options}
\NormalTok{mkt_vol[(}\KeywordTok{nrow}\NormalTok{(Market}\OperatorTok{$}\NormalTok{calls)}\OperatorTok{+}\DecValTok{1}\NormalTok{)}\OperatorTok{:}\NormalTok{nb_opts,}\DecValTok{3}\NormalTok{] <-}\StringTok{ }\NormalTok{Market}\OperatorTok{$}\NormalTok{puts[,}\DecValTok{2}\NormalTok{]}

\CommentTok{# Implied volatility of call options}
\NormalTok{mkt_vol[}\DecValTok{1}\OperatorTok{:}\KeywordTok{nrow}\NormalTok{(Market}\OperatorTok{$}\NormalTok{calls),}\DecValTok{4}\NormalTok{] <-}\StringTok{ }\NormalTok{Market}\OperatorTok{$}\NormalTok{calls[,}\DecValTok{3}\NormalTok{]}

\CommentTok{# Implied volatility of put options}
\NormalTok{mkt_vol[(}\KeywordTok{nrow}\NormalTok{(Market}\OperatorTok{$}\NormalTok{calls)}\OperatorTok{+}\DecValTok{1}\NormalTok{)}\OperatorTok{:}\NormalTok{nb_opts,}\DecValTok{4}\NormalTok{] <-}\StringTok{ }\NormalTok{Market}\OperatorTok{$}\NormalTok{puts[,}\DecValTok{3}\NormalTok{]}

\CommentTok{# Set a vector of initial values of a1, a2, a3, and a4}
\NormalTok{x0 <-}\StringTok{ }\KeywordTok{c}\NormalTok{(}\FloatTok{0.2}\NormalTok{, }\DecValTok{1}\NormalTok{, }\DecValTok{1}\NormalTok{, }\FloatTok{0.1}\NormalTok{)}

\CommentTok{# Calibrate the volatility surface on traded call and put options}
\NormalTok{tmp <-}\StringTok{ }\KeywordTok{optim}\NormalTok{(}\DataTypeTok{par =}\NormalTok{ x0,}
             \DataTypeTok{fn =}\NormalTok{ vol_calibrate)}

\CommentTok{# Store parameters in 'theta_vol'}
\NormalTok{theta_vol <-}\StringTok{ }\NormalTok{tmp}\OperatorTok{$}\NormalTok{par}

\CommentTok{# Generate a sequence of strike price and time to expiry}
\NormalTok{x1 <-}\StringTok{ }\NormalTok{sp_}\DecValTok{500}\NormalTok{[n_obs] }\OperatorTok{*}\StringTok{ }\KeywordTok{seq}\NormalTok{(}\FloatTok{0.5}\NormalTok{, }\FloatTok{1.5}\NormalTok{, (}\FloatTok{1.5} \OperatorTok{-}\StringTok{ }\FloatTok{0.5}\NormalTok{) }\OperatorTok{/}\StringTok{ }\DecValTok{1000}\NormalTok{)}
\NormalTok{x2 <-}\StringTok{ }\KeywordTok{seq}\NormalTok{(}\FloatTok{0.01}\NormalTok{, }\DecValTok{2}\NormalTok{, (}\DecValTok{2} \OperatorTok{-}\StringTok{ }\FloatTok{0.01}\NormalTok{) }\OperatorTok{/}\StringTok{ }\DecValTok{1000}\NormalTok{)}

\CommentTok{# Generate al possible combinations of 'x1' and 'x2'}
\NormalTok{x3 <-}\StringTok{ }\KeywordTok{expand.grid}\NormalTok{(x1,x2)}

\CommentTok{# Compute the implied volatility for each combination}
\NormalTok{y  <-}\StringTok{ }\KeywordTok{vol_surface}\NormalTok{(sp_}\DecValTok{500}\NormalTok{[n_obs], x3[,}\DecValTok{1}\NormalTok{], x3[,}\DecValTok{2}\NormalTok{], theta_vol[}\DecValTok{1}\NormalTok{], theta_vol[}\DecValTok{2}\NormalTok{], theta_vol[}\DecValTok{3}\NormalTok{], theta_vol[}\DecValTok{4}\NormalTok{])}

\CommentTok{# Create a 3D-plot of the fitted volatility surface}
\KeywordTok{plot3d}\NormalTok{(x3[,}\DecValTok{1}\NormalTok{], x3[,}\DecValTok{2}\NormalTok{], y)}
\end{Highlighting}
\end{Shaded}

\hypertarget{part-viii-full-approach}{%
\section{Part VIII: Full approach}\label{part-viii-full-approach}}

\begin{Shaded}
\begin{Highlighting}[]
\CommentTok{#install.packages("rugarch")}
\KeywordTok{library}\NormalTok{(}\StringTok{"rugarch"}\NormalTok{)}
\end{Highlighting}
\end{Shaded}

\begin{verbatim}
## Loading required package: parallel
\end{verbatim}

\begin{verbatim}
## Registered S3 method overwritten by 'xts':
##   method     from
##   as.zoo.xts zoo
\end{verbatim}

\begin{verbatim}
## 
## Attaching package: 'rugarch'
\end{verbatim}

\begin{verbatim}
## The following object is masked from 'package:stats':
## 
##     sigma
\end{verbatim}

\begin{Shaded}
\begin{Highlighting}[]
\CommentTok{# Residuals of the log-returns of the underlying using a Garch(1,1) with Normal innovations}
\NormalTok{spec   <-}\StringTok{ }\KeywordTok{ugarchspec}\NormalTok{(}\DataTypeTok{variance.model =} \KeywordTok{list}\NormalTok{(}\DataTypeTok{model =} \StringTok{"sGARCH"}\NormalTok{),}
                     \DataTypeTok{mean.model =} \KeywordTok{list}\NormalTok{(}\DataTypeTok{armaOrder =} \KeywordTok{c}\NormalTok{(}\DecValTok{0}\NormalTok{,}\DecValTok{0}\NormalTok{),}
                     \DataTypeTok{include.mean =} \OtherTok{FALSE}\NormalTok{))}

\NormalTok{fit           <-}\StringTok{ }\KeywordTok{ugarchfit}\NormalTok{(}\DataTypeTok{spec =}\NormalTok{ spec, }\DataTypeTok{data =}\NormalTok{ log_return)}
\NormalTok{Resid_returns <-}\StringTok{ }\NormalTok{fit}\OperatorTok{@}\NormalTok{fit}\OperatorTok{$}\NormalTok{residuals}

\CommentTok{# Residuals of the log-returns of the Vix using an AR(1) model }
\NormalTok{ar1_vix   <-}\StringTok{ }\KeywordTok{arima}\NormalTok{(vix_return,}\DataTypeTok{order =} \KeywordTok{c}\NormalTok{(}\DecValTok{1}\NormalTok{,}\DecValTok{0}\NormalTok{,}\DecValTok{0}\NormalTok{))}
\NormalTok{Resid_vix <-}\StringTok{ }\NormalTok{ar1_vix}\OperatorTok{$}\NormalTok{residuals}

\CommentTok{# Fit normal marginals by MLE}
\NormalTok{fit1 <-}\StringTok{ }\KeywordTok{suppressWarnings}\NormalTok{(}\KeywordTok{fitdistr}\NormalTok{(}\DataTypeTok{x =}\NormalTok{ Resid_returns,}
                         \DataTypeTok{densfun =}\NormalTok{ dnorm,}
                         \DataTypeTok{start =} \KeywordTok{list}\NormalTok{(}\DataTypeTok{mean =} \DecValTok{0}\NormalTok{, }\DataTypeTok{sd =} \DecValTok{1}\NormalTok{)))}
\NormalTok{theta1 <-}\StringTok{ }\NormalTok{fit1}\OperatorTok{$}\NormalTok{estimate}

\NormalTok{fit2 <-}\StringTok{ }\KeywordTok{suppressWarnings}\NormalTok{(}\KeywordTok{fitdistr}\NormalTok{(}\DataTypeTok{x =}\NormalTok{ Resid_vix,}
                         \DataTypeTok{densfun =}\NormalTok{ dnorm,}
                         \DataTypeTok{start =} \KeywordTok{list}\NormalTok{(}\DataTypeTok{mean =} \DecValTok{0}\NormalTok{, }\DataTypeTok{sd =} \DecValTok{1}\NormalTok{)))}
\NormalTok{theta2 <-}\StringTok{ }\NormalTok{fit2}\OperatorTok{$}\NormalTok{estimate}

\CommentTok{# Compute 'U_1' and 'U_2' and combine these two variables in 'U'}
\NormalTok{U1 <-}\StringTok{ }\KeywordTok{pnorm}\NormalTok{(Resid_returns, }\DataTypeTok{mean =}\NormalTok{ theta1[}\DecValTok{1}\NormalTok{], }\DataTypeTok{sd =}\NormalTok{ theta1[}\DecValTok{2}\NormalTok{])}
\NormalTok{U2 <-}\StringTok{ }\KeywordTok{pnorm}\NormalTok{(Resid_vix, }\DataTypeTok{mean =}\NormalTok{ theta2[}\DecValTok{1}\NormalTok{], }\DataTypeTok{sd =}\NormalTok{ theta2[}\DecValTok{2}\NormalTok{])}
\NormalTok{U  <-}\StringTok{ }\KeywordTok{cbind}\NormalTok{(U1, U2)}

\CommentTok{# Calibrate a Gaussian copula}
\NormalTok{C   <-}\StringTok{ }\KeywordTok{normalCopula}\NormalTok{(}\DataTypeTok{dim =} \DecValTok{2}\NormalTok{)}
\NormalTok{fit <-}\StringTok{ }\KeywordTok{fitCopula}\NormalTok{(C, }\DataTypeTok{data =}\NormalTok{ U, }\DataTypeTok{method =} \StringTok{"ml"}\NormalTok{)}

\CommentTok{# Set seed for generating pseudo-random numbers}
\KeywordTok{set.seed}\NormalTok{(}\DecValTok{4321}\NormalTok{)}
\NormalTok{sim_U             <-}\StringTok{ }\KeywordTok{rCopula}\NormalTok{(H }\OperatorTok{*}\StringTok{ }\NormalTok{t, fit}\OperatorTok{@}\NormalTok{copula)}
\NormalTok{sim_Resid_returns <-}\StringTok{ }\KeywordTok{qnorm}\NormalTok{(sim_U[,}\DecValTok{1}\NormalTok{], }\DataTypeTok{mean =}\NormalTok{ theta1[}\DecValTok{1}\NormalTok{], }\DataTypeTok{sd =}\NormalTok{ theta1[}\DecValTok{2}\NormalTok{])}
\NormalTok{sim_Resid_vix     <-}\StringTok{ }\KeywordTok{qnorm}\NormalTok{(sim_U[,}\DecValTok{2}\NormalTok{], }\DataTypeTok{mean =}\NormalTok{ theta2[}\DecValTok{1}\NormalTok{], }\DataTypeTok{sd =}\NormalTok{ theta2[}\DecValTok{2}\NormalTok{])}

\CommentTok{# Initialize the array 'sim_ret_4'}
\NormalTok{sim_ret_}\DecValTok{4}\NormalTok{ <-}\StringTok{ }\KeywordTok{array}\NormalTok{(}\DataTypeTok{data =} \OtherTok{NA}\NormalTok{, }\DataTypeTok{dim =} \KeywordTok{c}\NormalTok{(H, }\DecValTok{2}\NormalTok{, t))}

\CommentTok{# Store in 'sim_ret_4' daily residuals of log returns for the underlying asset and the VIX}
\ControlFlowTok{for}\NormalTok{ (i }\ControlFlowTok{in} \DecValTok{1}\OperatorTok{:}\NormalTok{t) \{}
\NormalTok{  sim_ret_}\DecValTok{4}\NormalTok{[,,i] <-}\StringTok{ }\KeywordTok{c}\NormalTok{(sim_Resid_returns[(H }\OperatorTok{*}\StringTok{ }\NormalTok{(i }\OperatorTok{-}\StringTok{ }\DecValTok{1}\NormalTok{) }\OperatorTok{+}\StringTok{ }\DecValTok{1}\NormalTok{)}\OperatorTok{:}\NormalTok{(H }\OperatorTok{*}\StringTok{ }\NormalTok{i)], sim_Resid_vix[(H }\OperatorTok{*}\StringTok{ }\NormalTok{(i }\OperatorTok{-}\StringTok{ }\DecValTok{1}\NormalTok{) }\OperatorTok{+}\StringTok{ }\DecValTok{1}\NormalTok{)}\OperatorTok{:}\NormalTok{(H }\OperatorTok{*}\StringTok{ }\NormalTok{i)])}
\NormalTok{\}}

\CommentTok{# Initialize two vectors that contain the residuals of the underlying asset price and the VIX value one week from now}
\NormalTok{S_}\DecValTok{5}\NormalTok{   <-}\StringTok{ }\KeywordTok{rep}\NormalTok{(}\OtherTok{NA}\NormalTok{, H)}
\NormalTok{vol_}\DecValTok{5}\NormalTok{ <-}\StringTok{ }\KeywordTok{rep}\NormalTok{(}\OtherTok{NA}\NormalTok{, H)}

\CommentTok{# Compute the underlying asset price and the VIX value one week from now}
\ControlFlowTok{for}\NormalTok{ (i }\ControlFlowTok{in} \DecValTok{1}\OperatorTok{:}\NormalTok{H) \{}
\NormalTok{  S_}\DecValTok{5}\NormalTok{[i]   <-}\StringTok{ }\NormalTok{S_}\DecValTok{1} \OperatorTok{*}\StringTok{ }\KeywordTok{exp}\NormalTok{(}\KeywordTok{sum}\NormalTok{(sim_ret_}\DecValTok{4}\NormalTok{[i,}\DecValTok{1}\NormalTok{,]))}
\NormalTok{  vol_}\DecValTok{5}\NormalTok{[i] <-}\StringTok{ }\NormalTok{vol_}\DecValTok{1} \OperatorTok{*}\StringTok{ }\KeywordTok{exp}\NormalTok{(}\KeywordTok{sum}\NormalTok{(sim_ret_}\DecValTok{4}\NormalTok{[i,}\DecValTok{2}\NormalTok{,]))}
\NormalTok{\}}

\CommentTok{# Initialize a matrix to store call prices (10 000 rows (1 per simulation), 4 columns (1 per option))}
\NormalTok{call_price_}\DecValTok{5}\NormalTok{ <-}\StringTok{ }\KeywordTok{matrix}\NormalTok{(}\OtherTok{NA}\NormalTok{, }\DataTypeTok{nrow =}\NormalTok{ H, }\DataTypeTok{ncol =} \DecValTok{4}\NormalTok{)}

\CommentTok{# Loop through 10 000 simulations and price each call option}
\ControlFlowTok{for}\NormalTok{ (i }\ControlFlowTok{in} \DecValTok{1}\OperatorTok{:}\DecValTok{10000}\NormalTok{)\{}
\NormalTok{  call_price_}\DecValTok{5}\NormalTok{[i,}\DecValTok{1}\NormalTok{] <-}\StringTok{ }\KeywordTok{price_call}\NormalTok{(S_}\DecValTok{5}\NormalTok{[i], K[}\DecValTok{1}\NormalTok{], r_}\DecValTok{3}\NormalTok{, vol_}\DecValTok{5}\NormalTok{[i], M[}\DecValTok{1}\NormalTok{] }\OperatorTok{-}\StringTok{ }\NormalTok{t }\OperatorTok{/}\StringTok{ }\DecValTok{250}\NormalTok{)}
\NormalTok{  call_price_}\DecValTok{5}\NormalTok{[i,}\DecValTok{2}\NormalTok{] <-}\StringTok{ }\KeywordTok{price_call}\NormalTok{(S_}\DecValTok{5}\NormalTok{[i], K[}\DecValTok{2}\NormalTok{], r_}\DecValTok{3}\NormalTok{, vol_}\DecValTok{5}\NormalTok{[i], M[}\DecValTok{2}\NormalTok{] }\OperatorTok{-}\StringTok{ }\NormalTok{t }\OperatorTok{/}\StringTok{ }\DecValTok{250}\NormalTok{)}
\NormalTok{  call_price_}\DecValTok{5}\NormalTok{[i,}\DecValTok{3}\NormalTok{] <-}\StringTok{ }\KeywordTok{price_call}\NormalTok{(S_}\DecValTok{5}\NormalTok{[i], K[}\DecValTok{3}\NormalTok{], r_}\DecValTok{4}\NormalTok{, vol_}\DecValTok{5}\NormalTok{[i], M[}\DecValTok{3}\NormalTok{] }\OperatorTok{-}\StringTok{ }\NormalTok{t }\OperatorTok{/}\StringTok{ }\DecValTok{250}\NormalTok{)}
\NormalTok{  call_price_}\DecValTok{5}\NormalTok{[i,}\DecValTok{4}\NormalTok{] <-}\StringTok{ }\KeywordTok{price_call}\NormalTok{(S_}\DecValTok{5}\NormalTok{[i], K[}\DecValTok{4}\NormalTok{], r_}\DecValTok{4}\NormalTok{, vol_}\DecValTok{5}\NormalTok{[i], M[}\DecValTok{4}\NormalTok{] }\OperatorTok{-}\StringTok{ }\NormalTok{t }\OperatorTok{/}\StringTok{ }\DecValTok{250}\NormalTok{)}
\NormalTok{\}}

\CommentTok{# Compute the price of the portfolio for each replication and store it in 'PF_price_5'}
\NormalTok{PF_price_}\DecValTok{5}\NormalTok{ <-}\StringTok{ }\KeywordTok{rowSums}\NormalTok{(call_price_}\DecValTok{5}\NormalTok{)}

\CommentTok{# Compute the P&L}
\NormalTok{PL_}\DecValTok{4}\NormalTok{ <-}\StringTok{ }\NormalTok{PF_price_}\DecValTok{5} \OperatorTok{*}\StringTok{ }\KeywordTok{exp}\NormalTok{(}\OperatorTok{-}\NormalTok{(t }\OperatorTok{/}\StringTok{ }\DecValTok{360}\NormalTok{) }\OperatorTok{*}\StringTok{ }\NormalTok{r_PL) }\OperatorTok{-}\StringTok{ }\NormalTok{PF_price_}\DecValTok{1}

\CommentTok{# Compute the VaR and the ES of the P&L distribution}
\NormalTok{VaR_}\DecValTok{4}\NormalTok{ <-}\StringTok{ }\KeywordTok{sort}\NormalTok{(PL_}\DecValTok{4}\NormalTok{)[(}\DecValTok{1} \OperatorTok{-}\StringTok{ }\NormalTok{alpha) }\OperatorTok{*}\StringTok{ }\NormalTok{H]}
\NormalTok{ES_}\DecValTok{4}\NormalTok{  <-}\StringTok{ }\KeywordTok{mean}\NormalTok{(}\KeywordTok{sort}\NormalTok{(PL_}\DecValTok{4}\NormalTok{)[}\DecValTok{1}\OperatorTok{:}\NormalTok{((}\DecValTok{1} \OperatorTok{-}\StringTok{ }\NormalTok{alpha) }\OperatorTok{*}\StringTok{ }\NormalTok{H)])}

\CommentTok{# Plot an histogram}
\KeywordTok{hist}\NormalTok{(PL_}\DecValTok{4}\NormalTok{, }\DataTypeTok{nclass =} \KeywordTok{round}\NormalTok{(}\DecValTok{10} \OperatorTok{*}\StringTok{ }\KeywordTok{log}\NormalTok{(}\KeywordTok{length}\NormalTok{(PL_}\DecValTok{4}\NormalTok{))), }\DataTypeTok{probability =} \OtherTok{TRUE}\NormalTok{)}

\CommentTok{# Add a vertical line to show the VaR}
\KeywordTok{abline}\NormalTok{(}\DataTypeTok{v   =} \KeywordTok{quantile}\NormalTok{(PL_}\DecValTok{4}\NormalTok{, }\DataTypeTok{probs =}\NormalTok{ (}\DecValTok{1} \OperatorTok{-}\StringTok{ }\NormalTok{alpha)),}
       \DataTypeTok{lty =} \DecValTok{1}\NormalTok{,}
       \DataTypeTok{lwd =} \FloatTok{2.5}\NormalTok{,}
       \DataTypeTok{col =} \StringTok{"red"}\NormalTok{)}
\end{Highlighting}
\end{Shaded}

\includegraphics{Project_Main_Notebook_files/figure-latex/unnamed-chunk-17-1.pdf}


\end{document}
